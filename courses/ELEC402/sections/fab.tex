\section{Fabrication of ICs}

\subsection{Digital Circuits}

\subsubsection{Packaging}
\begin{itemize}
    \item Packaged or covered with epoxy resin (plastic) or ceramic
    \item Wire bonding: connect the chip to the package
    \item PADS: connect the package to the PCB
\end{itemize}

IO buffer: connect the chip to the package\begin{itemize}
    \item change the voltage level
    \item remove noise
    \item improve rise/fall time
    \item protect the chip from ESD
    \item provide a constant current source
\end{itemize}

Driving PADs: \begin{itemize}
    \item connect the package to the PCB
    \item provide a constant current source
    \item remove noise
    \item protect the chip from ESD
\end{itemize}

\subsection{Wafer Fabrication}

Built in a clean room.

Wafer: a thin slice of semiconductor material.\begin{itemize}
    \item Silicon
    \item Gallium Arsenide
    \item Silicon Carbide
\end{itemize}

Growing a Silicon Crystal: \begin{itemize}
    \item Czochralski process
    \item Float zone process
    \item Epitaxial growth
\end{itemize}

\subsection{MOSFET}

MOSFET (Metal Oxide Semiconductor Field Effect Transistor): \begin{itemize}
    \item NMOS: n-channel MOSFET
    \item PMOS: p-channel MOSFET
    \item CMOS: complementary MOSFET
\end{itemize}

\subsection{Lithography}

\begin{itemize}
    \item Photolithography: use light to transfer a geometric pattern from a photomask to a light-sensitive chemical (photoresist) on the substrate.
    \item Etching: remove the unwanted material
    \item Ion implantation: change the electrical properties of the material
\end{itemize}

\subsubsection{N/P-Well CMOS Process}

