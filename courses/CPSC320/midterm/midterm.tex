\documentclass[conference]{IEEEtran}
\usepackage{cite}
\usepackage{amsmath,amssymb,amsfonts}
\usepackage{graphicx}
\usepackage{hyperref}
\usepackage{textcomp}
\usepackage{xcolor}
\usepackage{algorithm}
\usepackage{algorithmic}
\usepackage{amsmath, amsfonts, amssymb, ragged2e}
\usepackage{amsthm}
% \usepackage{algpseudocode}
% \usepackage{comment}



\newtheorem*{theorem}{Theorem}
\newtheorem*{definition}{Definition}
\newtheorem*{corollary}{Corollary}
\newtheorem*{lemma}{Lemma}
\newtheorem*{proposition}{Proposition}
\def\BibTeX{{\rm B\kern-.05em{\sc i\kern-.025em b}\kern-.08em
T\kern-.1667em\lower.7ex\hbox{E}\kern-.125emX}}
\setcounter{MaxMatrixCols}{12} % increase maximum matrix width
\begin{document}


\section{Geometric Series}
\begin{align*}
    \sum_{k=0}a r^k &= a (\frac{1-r^{n+1}}{1-r})\\
    & = \frac{a}{1-r} \forall |r| < 1
\end{align*}
\section{SMP}
It is always true that employers with the same preferences will have the same stable matching.\begin{proof}
    Prove inductively, remove the most stable pair, then the remaining employers and employees have the same preferences. Continue until all pairs are removed.
\end{proof}
\section{Asymptotic}
log $<$ polynomial $<$ exponential $<$ factorial
\section{Greedy Algorithms}
Chase only the local optimum, not the global optimum.
\subsection{Greedy stays ahead}

    If a greedy algorithm always makes a choice that stays ahead (at least as good) of the optimal solution, then the greedy algorithm is optimal.

    Inductively show that at each stage, the greedy solution is at least as good as the optimal solution.

\section{Divide and Conquer}
Divide the input into smaller instances (subproblems), solve recursively, then combine to get the solution.

\subsection{Recurrence Relations}
\subsubsection{Recurrence Tree}
Each level of the tree represents the cost of the work done at that level. The total cost is the sum of the costs at each level. Usually $\log n$ levels with a $f(n)$ cost at each level.
\subsubsection{Master Theorem}
Let \underline{$a \geq 1$ and $b > 1$} be constants, let $f(n)$ be a function, and let $T(n)$ be defined on the non-negative integers by the recurrence relation: \[
    T(n) = aT(n/b) + f(n)
\]
where $n/b$ means either $\lfloor n/b \rfloor$ or $\lceil n/b \rceil$. Then $T(n)$ has the following asymptotic bounds: \begin{enumerate}
    \item If $f(n) = O(n^{\log_b a - \epsilon})$ for some constant $\epsilon > 0$, then $T(n) = \Theta(n^{\log_b a})$.
    \item If $f(n) = \Theta(n^{\log_b a} \log^k n)$, then $T(n) = \Theta(n^{\log_b a} \log^{k+1} n)$. Note the \textbf{+1} in the exponent.
    \item If $f(n) = \Omega(n^{\log_b a + \epsilon})$ for some constant $\epsilon > 0$, and if $a f(n/b) \leq k f(n)$ for some constant $k < 1$ and sufficiently large $n$, then $T(n) = \Theta(f(n))$.
\end{enumerate}

\subsection{Prune and Search}

\begin{definition}
    \textbf{Prune and Search} is a problem-solving strategy that divides the search space into two parts, one of which is pruned (discarded) and the other is searched.
\end{definition}

To find the $k$th smallest element in an array, we can use the QuickSelect algorithm. The algorithm is as follows:
\begin{enumerate}
    \item Function{QuickSelect}{$(A[1:n], k)$}
    \item If $n = 1$ return $A[1]$
    \item Choose a random pivot element $p$ from $A$
    \item Partition $A$ into $L = \{x \in A : x < p\}$, $E = \{x \in A : x = p\}$, $G = \{x \in A : x > p\}$
    \item If $k \leq |L|$, return QuickSelect$(L, k)$
    \item Else if $k \leq |L| + |E|$, return $p$
    \item Else return QuickSelect$(G, k - |L| - |E|)$
    \item End Function
\end{enumerate}
\section{Dynamic Programming}

Typical DP examples: \[
    \text{DP}[i] = \begin{cases}
        \text{base case} & \text{start} \\
        \text{combine}(\text{DP}[i-1], \text{DP}[i-2], \ldots, \text{DP}[i-k]) & \text{otherwise}
    \end{cases}
\]

Or \[
    \text{OPT}(j) = \max \{ \text{OPT}(j-1), \text{OPT}(j-2) + v_j \}
\]
Iterative starts with the base case and builds up to the solution. Recursive starts with the solution and breaks it down to the base case.

\section{NP-Completeness}
As long as the algorithm runs in $O(n^k)$ time, it is considered polynomial time. We think it is efficient.

\subsection{Decision V.S. Optimization}
\begin{definition}
    Optimization problem: we want to find the solution $s$ that maximizes or minimizes some objective function $f(s)$.
\end{definition}
\begin{definition}
    Decision problem: given a parameter $k$, we want to know if there is a solution $s$ such that $f(s) \geq k$ (maximization) or $f(s) \leq k$ (minimization).
\end{definition}
\subsection{P and NP}
\begin{definition}
    \textbf{P} is the class of decision problems that can be solved in polynomial time. (We have the solver)
\end{definition}
\begin{definition}
    \textbf{NP} is the class of decision problems for which a solution can be verified in polynomial time. (We have the verifier)
\end{definition}

The NP-complete problems are the NP problems with the property that, if we can solve this problem in polynomial time, we can solve any problem in NP in polynomial time. 
\subsection{Proof of NP-completeness}
\begin{enumerate}
    \item Show that the problem is in NP.
    \item Show that the problem is in NP-hard.
\end{enumerate}
\subsubsection{Proof of NP}
To prove that P is in NP, we need to \textbf{efficiently verify} a \textbf{certificate}.\begin{itemize}
    \item A certificate is a proof that the solution is correct.
    \item A verifier is an algorithm that checks the certificate. Needs to run in polynomial time.
    \item For example, an optimization problem can be: Does there exists $X$ such that $Y$ is true?\begin{itemize}
        \item $X$ is the certificate.
        \item $Y$ is the verifier.
    \end{itemize}
\end{itemize}
\subsubsection{Proof of NP-hard}
\begin{definition}
    A problem $A$ is \textbf{NP-hard} if $P$ is at least as hard as any other problem in NP.
\end{definition}
We prove ``at least as hard'' via polynomial-time reduction. 
\begin{itemize}
    \item Pick a known NP-complete problem $B$. Let $A$ be the problem we want to prove is NP-hard.
    \item Show that $B$ can be reduced to $A$ in polynomial time with the same YES/NO answer.
    \item Since $B$ is NP-complete, $A$ is NP-hard.
    \item Intuition: Since we can use a solver for $A$ to solve $B$, this tells us that $A$ is at least as hard as $B$ (It is powerful enough to handle $B$)
\end{itemize}


% \begin{algorithmic}
%     \Function{QuickSelect}{$A[1:n], k$} \Comment{ Returns the element of rank k in an array of n numbers}
%     \If {n = 1}
%         \State \Return A[1]
%     \EndIf
%     \State Choose a random pivot element p from A
%     \State Partition A into $L = \{x \in A : x < p\}$, $E = \{x \in A : x = p\}$, $G = \{x \in A : x > p\}$
%     \If {k $\leq |L|$}
%         \State \Return QuickSelect(L, k)
%     \ElsIf {k $\leq |L| + |E|$}
%         \State \Return p
%     \Else
%         \State \Return QuickSelect(G, k - |L| - |E|)
%     \EndIf
%     \EndFunction
% \end{algorithmic}

% \begin{algorithm}
%     \begin{algorithmic}
%         \Function{QuickSelect}{$A[1:n], k$} \Comment{ Returns the element of rank k in an array of n numbers}
%         \If {n = 1}
%             \State \Return A[1]
%         \EndIf
%         \State Choose a random pivot element p from A
%         \State Partition A into $L = \{x \in A : x < p\}$, $E = \{x \in A : x = p\}$, $G = \{x \in A : x > p\}$
%         \If {k $\leq |L|$}
%             \State \Return QuickSelect(L, k)
%         \ElsIf {k $\leq |L| + |E|$}
%             \State \Return p
%         \Else
%             \State \Return QuickSelect(G, k - |L| - |E|)
%         \EndIf
%         \EndFunction
%     \end{algorithmic}
%   \end{algorithm}
\section{Good to know}
\begin{itemize}
    \item $f(n) \in \theta(g(n)) \not \implies 2^{f(n)} \in \theta(2^{g(n)})$ 
    \item Spanning tree with two leaves is Hamiltonian cycle.
    \item Watch out for \textbf{All matchings/ set of possible answers}
    \item polynomial time $\times$ polynomial time = polynomial time. If $P\neq NP$, then not solvable in polynomial time.
    \item Recursion will have a \textbf{base case} and a \textbf{recursive case} where the problem is \textbf{broken down}.
    \item Reduction from A to B just needs all A instances to be transformed to some specific B instances.
    \item QuickSort: $O(n^2)$ worst case, $O(n \log n)$ average case. \begin{enumerate}
        \item Find a pivot element.
        \item Partition the array into two parts: elements less than the pivot and elements greater than the pivot.
        \item Recursively sort the two parts. Then combine.
    \end{enumerate}
    \item 4 SAT to 3 SAT: $(a \lor b \lor c \lor d) = (a \lor b \lor x) \land (\neg x \lor c \lor d)$
    \item To memorize a DP solution, use a 2D array.
    \item Recursive to iterative: watch out for $1\to n$ or $n\to 1$. Solve the base case first.
    \item Choose function: $n \choose k$ = $n-1 \choose k-1$ + $n-1 \choose k$$\in O(n^i)$. Some polynomials. 
    \item A memorized recursive solution can use a helper method which takes in the array and the index. If still the initialized value, then calculate it.
    \item If you can reduce a problem to another NP-complete problem, then it is in NP.
    \item Try not to set a reward/weight to 0. It might cause problems.
    \item inter means between, intra means within.
\end{itemize}
\subsection{Some NP-complete problems}
\subsubsection{Sequencing Problems}
\begin{enumerate}
    \item The traveling salesman problem (TSP): Given a list of cities and the distances between them, what is the shortest possible route that visits each city exactly once and returns to the origin city?
    \item The Hamiltonian cycle problem: Given a graph, does there exist a cycle that visits each vertex exactly once?
    \item The Hamiltonian path problem: Given a graph, does there exist a path that visits each vertex exactly once?
\end{enumerate}
\subsubsection{Partition Problems}
\begin{enumerate}
    \item The 3 Dimensional Matching problem: Given three sets $X, Y, Z$ each with $n$ elements, and a set $T \subseteq X \times Y \times Z$, is there a subset $T' \subseteq T$ such that $|T'| = n$ and no two elements in $T'$ share a common index?
    \item Max-Min Clustering Problem (MMCP): Given a set of points in a metric space, and a number $k$, is it possible to partition the points into $k$ clusters such that the minimum distance between any two points in the same cluster is maximized?
    \item Graph colouring (Not NP-complete, but 3-colouring is NP-complete): Given an undirected graph, is it possible to colour the vertices using at most $k$ colours such that no two adjacent vertices share the same colour?
\end{enumerate}
\subsubsection{Numerical Problems}
\begin{enumerate}
    \item The subset sum problem: Given a set of integers, is there a non-empty subset whose sum is zero?
    \item Schedule problems: Given a set of tasks, each with a start time, end time, and a weight, what is the maximum weight subset of tasks that can be scheduled without overlapping?
\end{enumerate}
\subsubsection{Packing Problems}
\begin{enumerate}
    \item Independent set: Given a graph, is there a set of vertices such that no two vertices are adjacent?
    \item Set Packing: Given a set of sets, is there a subset of the sets such that no two sets share a common element?
\end{enumerate}
\subsubsection{Covering Problems}
\begin{enumerate}
    \item Hiiting Set: Given a set of sets, is there a subset of the sets such that every element is covered at least once?
    \item Dominating Set: Given a graph, is there a set of vertices such that every vertex is either in the set or adjacent to a vertex in the set?
    \item Vertex Cover: Given a graph, is there a set of vertices such that every edge is incident to at least one vertex in the set?
    \item Set Cover: Given a set of sets, is there a subset of the sets such that every element is covered at least once?
\end{enumerate}
\end{document}