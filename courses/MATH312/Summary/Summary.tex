\documentclass[letterpaper,12pt,oneside]{article}
\setlength{\headheight}{14.49998pt}
\usepackage{fancyhdr}
\usepackage{lipsum,graphicx}
\usepackage{amsmath, amsfonts, amssymb, ragged2e}
\usepackage{amsthm}
\usepackage{bookmark}
\usepackage{listings}
\usepackage{times}

\newtheorem{theorem}{Theorem}
\newtheorem{definition}{Definition}
\newtheorem*{corollary}{Corollary}
\newtheorem*{lemma}{Lemma}
\newtheorem*{proposition}{Proposition}



\title{Summary of Math 312}
\author{Tom Wang}
\date{Spring, 2024}

\fancypagestyle{plain}{
    \fancyhf{}
    \fancyhead[L]{Tom Wang}
    \fancyhead[R]{\thepage}
}

\begin{document}

\maketitle
\thispagestyle{plain}
\section{Induction}
Some basic properties of the natural numbers:
\begin{itemize}
    \item Mathematical induction

          Suppose $S \subseteq \mathbb{N}$ such that:\begin{itemize}
              \item $1 \in S$
              \item $\forall n \in \mathbb{N}, n \in S \implies n+1 \in S$
          \end{itemize}
    \item Strong Induction

          Suppose $S \subseteq \mathbb{N}$ such that:\begin{itemize}
              \item $1 \in S$
              \item $\forall n \in \mathbb{N}, \forall k \in \mathbb{N}, 1 \leq k \leq n \implies k \in S \implies n+1 \in S$
          \end{itemize}
    \item Well-Ordering Principle: Every non-empty subset of $\mathbb{N}$ has a least
          element.
\end{itemize}

Note that the above three are equivalent.

\subsection{Examples}
Let $\alpha = \frac{1+\sqrt{5}}{2}, \beta = \frac{1-\sqrt{5}}{2}$, Prove that
the n-th Fibonacci number $F_n = \frac{\alpha^n - \beta^n}{\alpha - \beta} =
    \frac{\alpha^n - \beta^n}{\sqrt{5}}$.
\begin{proof}
    Note that $\alpha, \beta$ are the roots of $x^2 - x - 1 = 0$. Then we have $\alpha^2 = \alpha + 1, \beta^2 = \beta + 1$. We will prove the statement by \textbf{Strong} induction on $n$.

    \textbf{Base Case:} $n = 1, F_1 = 1 = \frac{\alpha - \beta}{\sqrt{5}} = \frac{\alpha^1 - \beta^1}{\sqrt{5}}$.

    $n=2, F_2 = 1 = \frac{\alpha^2 - \beta^2}{\sqrt{5}}$.

    \textbf{Inductive Step:} Suppose $F_k = \frac{\alpha^k - \beta^k}{\sqrt{5}}$ for all $k \leq n$. Then we have $F_{n+1} = F_n + F_{n-1} = \frac{\alpha^n - \beta^n}{\sqrt{5}} + \frac{\alpha^{n-1} - \beta^{n-1}}{\sqrt{5}} = \frac{\alpha^{n-1}(\alpha + 1) - \beta^{n-1}(\beta + 1)}{\sqrt{5}} = \frac{\alpha^{n+1} - \beta^{n+1}}{\sqrt{5}}$.

    Used the properties of $\alpha, \beta$ in the last step.
\end{proof}

Another example:

Prove that all positive integers are a product of primes. Here we treat the
empty product as 1 and the product of one prime as itself.

\begin{proof}
    We use proof by contradiction to show the use of the well-ordering principle.

    Let S be the set of positive integers which cannot be written as products of
    primes. \textbf{Assume} that S is not empty:

    Then by the well-ordering principle, S has a least element $n$. Since $n$ is
    the least element, it cannot be prime.

    Then $n = ab$ for some $a, b \in \mathbb{N}$. Then $a, b < n$. Since $n$ is the
    least element, $a, b \notin S$. Then $a, b$ can be written as products of
    primes. Then $n = ab$ can be written as a product of primes. This is a
    contradiction.

    Therefore, S is empty and all positive integers can be written as products of
    primes.
\end{proof}
\subsection{The division algorithm}
Let $a\in \mathbb{Z}, a\ge 1$. $\forall n \in \mathbb{Z}, \exists q, r \in
    \mathbb{Z}$ such that $n = aq + r$ and $0 \leq r < a$. Moreover, $q, r$ are
unique.

\begin{proof}
    Prove using the well-ordering principle.

    Let $S = \{s \in \mathbb{Z} | s\ge 0 \land s=n-aq \land a,q,n \in
        \mathbb{Z}\}$. We claim that $S$ is not empty.

    Indeed $n-aq\ge0 \iff n\ge aq \iff q\le \frac{n}{a}$, given $a>0$. So there are
    infinitely many $q$ that satisfy the condition, which proves that $S$ is not
    empty.

    By the well-ordering principle, $S$ has a least element $r$. Then $r = n-aq$
    for some $q$. Then $n = aq + r$. It remains to show that $0 \leq r < a$. We
    argue by contradiction.

    If $r \ge a$, then $r-a \ge 0$. Then $r-a = n-aq-a = n-a(q+1) \in S$. But $r-a
        < r$, which contradicts the fact that $r$ is the least element of $S$.
    Therefore, $r < a$.

    To show uniqueness, suppose $n = aq + r = aq' + r'$, where $0 \leq r, r' < a$.
    Our goal is to show that $q = q'$ and $r = r'$.

    Subtracting the two equations, we have $a(q-q') = r' - r$. Notice that the LHS
    is divisible by $a$ and the RHS is less than $a$ but greater than 0. Therefore,
    $r' - r = 0$ and $q-q' = 0$. Therefore, $q = q'$ and $r = r'$.

    This completes the proof.
\end{proof}

Remark: q stands for \textbf{quotient} and r stands for \textbf{remainder}.

Also, the Division algorithm is a \textbf{theorem}. The actual algorithm to
find q and r is called \textbf{long division}.
\section{Primes}
\begin{definition}
    A positive integer $p$ is called \textbf{prime} if $p \ge 2$ and the only positive divisors of $p$ are 1 and $p$.
\end{definition}
\subsection{Euclid's Theorem}
\begin{theorem}
    There are infinitely many primes.
\end{theorem}
\begin{proof}
    Suppose there are only finitely many primes $p_1, p_2, \dots, p_n$.

    Let $N = p_1p_2\dots p_n + 1$. (Products of primes plus 1)

    Then $N$ is not divisible by any of the primes $p_1, p_2, \dots, p_n$.
    Therefore, $N$ is either prime or divisible by a prime larger than $p_n$.

    This is a contradiction. Therefore, there are infinitely many primes.
\end{proof}
The theorem can be used to generate infinitely many primes. For example, $2, 2\cdot 3 + 1 = 7, 2\cdot 3\cdot 5 + 1 = 31, \dots$ are all primes. However, this is not a good way to generate primes because the numbers get very large very quickly.

\subsection{Sieve of Eratosthenes}
\begin{theorem}
    Let $n \in \mathbb{N}, n \ge 2$. Then there exists a prime $p$ such that $p \leq n$.
\end{theorem}
The key idea is that if $n=ab$ and n is a composite, then at least one of $a, b$ is less than or equal to $\sqrt{n}$. So we only need to check up to $\sqrt{n}$ to see if $n$ is a prime.

For example, to check the primes within 25, we only need to check up to
$\sqrt{25} = 5$. List out all the numbers from 2 to 25. Then cross out all the
multiples of 2, 3, 5. The remaining numbers are all primes.

\subsection{The prime number theorem}
\begin{theorem}
    Let $\pi(x)$ be the number of primes less than or equal to $x$. Then $\lim_{x\to\infty} \frac{\pi(x)}{x/\ln x} = 1$.

    In other words, the number of primes less than or equal to $x$ is approximately
    $\frac{x}{\ln x}$.
\end{theorem}

\section{Division}
\subsection{Greatest common divisor}
\begin{definition}
    Let $a, b \in \mathbb{Z}, (a, b) \neq (0, 0)$. The \textbf{greatest common divisor} of $a$ and $b$, denoted by $\gcd(a, b)$, is the largest positive integer that divides both $a$ and $b$.
\end{definition}
Simple properties of gcd:\begin{itemize}
    \item $a\neq 0 \implies \gcd(a, 0) = |a|$.
    \item $\gcd(a, b)\ge 1$.
    \item $\gcd(a, b)=d \implies \gcd(\frac{a}{d},\frac{b}{d})=1$.
\end{itemize}
\begin{proof}
    \begin{itemize}
        \item Common divisors of $a$ and $0$ are all divisors of $a$. The largest one is
              $|a|$.
        \item We know 1 is a common divisor of $a$ and $b$. Therefore, $\gcd(a, b) \ge 1$.
        \item Suppose $d = gcd(a, b), e = gcd(\frac{a}{d}, \frac{b}{d})$.

              We write $\frac{a}{d} = ek, \frac{b}{d} = em$ for some $k, m \in \mathbb{Z}$.
              Then $a = (de)k, b = (de)m$. Now we can know that $de$ is a common divisor of
              $a$ and $b$. Thus $de \leq d$. Since $d \ge 1$, we have $e \leq 1$. Therefore,
              $e = 1$. Proof concluded.
    \end{itemize}
\end{proof}

\subsubsection{Some Crollaries}
\begin{enumerate}
    \item An integer $e$ is a common divisor of $a$ and $b$ if and only if $e | gcd(a,
              b)$.\begin{proof}

              If $e$ is a common divisor of $a$ and $b$, then $e | a$ and $e | b$. Therefore,
              $e | gcd(a, b) = ma + nb$ (Bezout's identity which will be proven below) for
              some $m, n \in \mathbb{Z}$. Therefore, $e | gcd(a, b)$.

              If $e | gcd(a, b)$, then $gcd(a, b) = ek$ for some $k \in \mathbb{Z}$. Then
              $gcd(a, b) = ek = ma + nb$. Therefore, $e | a$ and $e | b$. Therefore, $e$ is a
              common divisor of $a$ and $b$.
          \end{proof}
    \item Let $c$ be an integer, then $ax+by = c$ has an integer solution if and only if
          $gcd(a, b) | c$.\begin{proof}
              \item
              If $ax + by = c$ has an integer solution, then $gcd(a, b) | ax + by = c$.

              If $gcd(a, b) | c$, then $c = k(gcd(a, b)) = k(ax + by) = (ka)x + (kb)y$.
              Therefore, $ax + by = c$ has an integer solution.
          \end{proof}
\end{enumerate}

\subsection{Bezout's identity}
An integer linear combination of $a$ and $b$ is an integer of the form $ax +
    by$, where $x, y \in \mathbb{Z}$.
\begin{theorem}
    gcd(a, b) is the smallest positive integer that can be written as an integer linear combination of $a$ and $b$.
\end{theorem}
\begin{proof}
    Let S be the set of positive integers that can be written as an integer linear combination of $a$ and $b$. We want to show that $gcd(a, b)$ is the smallest element of S.

    Note that since $(a, b) \neq (0, 0)$, $S$ is not empty. By the well-ordering
    principle, $S$ has a least element $d$. We want to show that $d = gcd(a, b)$.
    To show it, we need to show that: \begin{itemize}
        \item $gcd(a ,b) | d$. This tells us that $gcd(a, b) \leq d$.
        \item $d | a$ and $d | b$. This tells us that $d \leq gcd(a, b)$.
    \end{itemize}

    By definition, $d=x_0a+y_0b$ for some $x_0, y_0 \in \mathbb{Z}$ which is a
    linear combination of a and b. But we know that $gcd(a, b) | (ma + nb),\forall
        m, n \in \mathbb{Z}$. Also, $d = x_0a + y_0b$. Therefore, $gcd(a, b) | d$.

    Now we want to show that $d | a$ and $d | b$. Let $a=dq+r$ where q is the
    quotient and r is the remainder. Just need to show that r = 0. Note that: \[
        r = a - dq = a - (x_0a + y_0b)q = a(1-x_0q) + b(-y_0q)
    \]
    So r is a linear combination of a and b. Since r is a remainder, $0 \leq r <
        d$. Since d is the smallest positive integer in S, r must be 0. Therefore, $d |
        a$. Similarly, $d | b$.
\end{proof}

\subsection{Euclidean Division Algorithm}
The Euclidean algorithm is a highly \textbf{recursive} and fast algorithm to
find the gcd of two numbers.

The complexity of the algorithm is $O(\log n)$, where $n$ is the larger of the
two numbers.

Before starting, we need to know the following lemma:

\subsubsection{Lemma}
Let $a, b, q, r \in \mathbb{Z}$ such that $a = bq + r$. Then $gcd(a, b) =
    gcd(b, r)$.\begin{proof}
    Let $d = gcd(a, b), d' = gcd(b, r)$. We want to show that $d = d'$.

    We know that $d' | b$ and $d' | r$. Therefore, $d' | bq + r = a$. Therefore,
    $d' | a$. So $d' \leq d$, it is a common divisor of $a$ and $b$.

    At the same time, $d | a$ and $d | b$. Therefore, $d | bq + r = a$. Therefore,
    $d | a$ and $d | r$. So $d \leq d'$, it is a common divisor of $b$ and $r$.

    Combine two results: $d = d'$.
\end{proof}
\subsubsection{Euclidean algorithm}
The algorithm applies the lemma repeatedly until we get $r = 0$ if we have
$a,b\ge 0$. Then the last non-zero remainder is the gcd of $a$ and $b$.

Give a recursion code example:\begin{lstlisting} [language=C++]
    int gcd(int a, int b) {
        if (b == 0) return a;
        if (a == 0) return b;
        if (a < b) return gcd(b % a, a);
        return gcd(b, a % b);
    }
\end{lstlisting}
So basically, we keep replacing $\gcd(a, b)$ with $\gcd(b, r)$ until $r = 0$ if
we choose $a\ge b$ the whole time. Then the last non-zero remainder is the gcd
of $a$ and $b$.

\subsubsection{Number of steps in Euclidean algorithm}
Lemma:

Let $a, b \in \mathbb{Z}, a \ge b \ge 1$. We let $a = bq + r$ where $0 \leq r <
    b$. Then $r < \frac{a}{2}$. \begin{proof}
    \begin{itemize}
        \item Case 1: $b \leq \frac{a}{2}$. Then $r < b \leq \frac{a}{2}$.
        \item Case 2: $b > \frac{a}{2}$. Then $r = a - bq < a - \frac{a}{2}q =
                  \frac{a}{2}(2-q) \leq \frac{a}{2}$. Given $q \ge 1$ since $a \ge b \le 0 \land
                  r \ge 0$.
    \end{itemize}
\end{proof}

Corollary:

Assume $a \ge b \ge 1$. Then the number of steps in the Euclidean algorithm is
at most $2 \log_2 a$. \begin{proof}
    By the lemma above, we know that for each iteration, $r < \frac{a}{2}$. Therefore, after $k$ iterations, $r < \frac{a}{2^k}$.

    But $r$ is a positive integer otherwise it terminates (The last r before
    becoming zero is the gcd, we have r = 0 to be the termination point in the
    sample code since it is easier to program).

    We want to find the smallest $k$ such that $\frac{a}{2^k} < 1$ (termination
    point). This is equivalent to $2^k > a$. Therefore, $k > \log_2 a$. Therefore,
    the number of iterations is at most $\log_2 a$.
\end{proof}

\subsubsection{Lame's theorem}
\begin{theorem}
    Suppose $a, b \in \mathbb{Z}, a \ge b \ge 1$. Let $d$ be the number of steps in the Euclidean algorithm. Then $d \leq 5 \log_{10} a$.
\end{theorem}

The proof can be found on Wikipedia. The $\log_{10}$ is used because the proof
involved Fibonacci numbers for some reason.

Now compare the bound by Lame's theorem and the bound by the corollary above.
We can compare the two bounds:\[
    2 \log_2 a = \frac{2}{\log_{10} 2} \log_{10} a \approx 6.64 \log_{10} a > 5 \log_{10} a
\]
So Lame's theorem is a better bound than the corollary above.

\subsection{Linear equations}
Solve linear equations in the form of $ax + by = c$ where $a, b, c \in
    \mathbb{Z}, (a, b)=(0, 0)$. And we are interested in the integer solutions of
$x, y$. Note that real solutions are easy to find and there are infinitely many
of them.

Bezout's identity proved that an integer solution exists if and only if
$\gcd(a, b) | c$. $\gcd(a, b)$ is the smallest positive integer that can be
written as an integer linear combination of $a$ and $b$. So all linear
combinations of $a$ and $b$ are multiples of $\gcd(a, b)$.

Now we want to find all the integer solutions.\begin{itemize}
    \item Describe the set of all integer solutions.
    \item Develop an algorithm to find all the integer solutions if exist
\end{itemize}

Let $d = \gcd(a,b)$. \begin{enumerate}
    \item if $c=d$, by Bezout's theorem, there exists $ax+by=d$ for some $x, y \in
              \mathbb{Z}$. To find them, use the Euclidean algorithm. Set $r_0 =a, r_1 = b$
          assuming $a \ge b \ge 0$. Then $r_0 = r_1q_1 + r_2$. Then $r_1 = r_2q_2 + r_3$.
          Keep going until $r_{n-1} = r_nq_n + 0$. Then $r_n = d$. Then we can find $x,
              y$ by back substitution.

          \textbf{Back substitution}: \begin{itemize}
              \item start from $d = r_{n+1}=r_{n-1}-r_nq_n=r_{n-1}-q_n(r_{n-2}-r_{n-1}q_{n-1})$ So
                    moving from a linear combination of $r_{n-1}$ and $r_n$ to a linear combination
                    of $r_{n-2}$ and $r_{n-1}$.
              \item continues until we get a linear combination of $r_0=a$ and $r_1=b$.
          \end{itemize}
          For example: find an integer solution for $154x+35y=7$. \begin{align*}
              154 & = 35 \cdot 4 + 14 \\
              35  & = 14 \cdot 2 + 7  \\
              14  & = 7 \cdot 2 + 0
          \end{align*}
          So $\gcd(154, 35) = 7$. Then we can find $x, y$ by back substitution.

          \begin{align*}
              7 & = 35 - 14 \cdot 2                 \\
                & = 35 - (154 - 35 \cdot 4) \cdot 2 \\
                & = 35 \cdot 9 - 154 \cdot 2
          \end{align*}
          So $x = -2, y = 9$ is a solution.

          Another example: $553x+327y=1$. \begin{align*}
              553 & = 327 \cdot 1 + 226 \\
              327 & = 226 \cdot 1 + 101 \\
              226 & = 101 \cdot 2 + 24  \\
              101 & = 24 \cdot 4 + 5    \\
              24  & = 5 \cdot 4 + 4     \\
              5   & = 4 \cdot 1 + 1     \\
              4   & = 1 \cdot 4 + 0
          \end{align*}
          So $\gcd(553, 327) = 1$. Then we can find $x, y$ by back substitution.
          \begin{align*}
              1 & = 5 - 4                                                                      \\
                & = 5 - (24 - 5 \cdot 4) = 5 \cdot 5  - 24 \cdot 1                             \\
                & = (101 - 24 \cdot 4) \cdot 5 - 24 \cdot 1 = 101 \cdot 5 - 24 \cdot 21        \\
                & = 101 \cdot 5 - (226 - 101 \cdot 2) \cdot 21 = 101 \cdot 47 - 226 \cdot 21   \\
                & = (327 - 226) \cdot 47 - 226 \cdot 21 = 327 \cdot 47 - 226 \cdot 68          \\
                & = 327 \cdot 47 - (553 - 327 \cdot 1) \cdot 68 = 327 \cdot 115 - 553 \cdot 68
          \end{align*}
          Now we can find $x = -68, y = 115$.

    \item Back to $ax+by=c$. Now assume that $d= \gcd(a,b) | c$. So $c = td$, for some
          $t\in \mathbb{Z}$ \begin{itemize}
              \item In this case, find a solution $(x_0, y_0)$ for $ax+by=d$.
              \item Then $x = x_0t, y = y_0t$ is a solution for $ax+by=c$.
              \item This is because $a(x_0t) + b(y_0t) = td = c$.
          \end{itemize}
    \item If $d \nmid c$, then there are no integer solutions.

          For example, $154x+35y=24$ has no integer solutions. This is because $24$ is
          not a multiple of $7$. $t$ is not an integer.

    \item Now we want to describe all solutions: \begin{theorem}
              Suppose $a,b,c \in \mathbb{Z}, (a,b) \neq (0,0)$.

              Let $d = \gcd(a,b)$. Then the set of all integer solutions of $ax+by=c$ is
              given by \[
                  \{(x_0+\frac{b}{d}t, y_0-\frac{a}{d}t) | t \in \mathbb{Z}\}
              \]
              if $d | c \iff c=td$. Otherwise, the set of all integer solutions is empty.
          \end{theorem}

          Note that $(x_0, y_0)$ is a particular solution of $ax+by=c$. Here ``general
          solution'' means that $(x,y)$ is a solution for every integer value of t, and
          every integer solution can be expressed in this form.
\end{enumerate}
\subsubsection{Key Lemma}\label{sec:key-lemma}
Before proving the theorem, we need to prove the following lemma:

\textbf{Key Lemma}: Let $a, b, c \in \mathbb{Z}, (a, b) \neq (0, 0)$. Let $d = \gcd(a, b) = 1$. Then $a|bc \implies a|c$.

Note that the lemma is not true if $d \neq 1$. \begin{proof}
    By Bezout's theorem, $\gcd(a,b)=1 \implies ax+by=1$ for some $x, y \in \mathbb{Z}$. Then multiply both sides by $c$: $acx+bcy=c$. Then $a|bc \implies a|acx+bcy=c$ as defined in the Lemma.
\end{proof}
\subsubsection{Proof of the general solution theorem}
With the Lemma in hand, we can start proving the theorem:\begin{proof}
    \begin{itemize}
        \item First need to show that $x=x_0+\frac{b}{d}t, y=y_0-\frac{a}{d}t$ is a solution
              of $ax+by=c, \forall t \in \mathbb{Z}$.

              We have $ax_0+by_0=ax+by=c$. Then \begin{align*}
                  ax+by & = a(x_0+\frac{b}{d}t) + b(y_0-\frac{a}{d}t)   \\
                        & = ax_0 + \frac{ab}{d}t + by_0 - \frac{ab}{d}t \\
                        & = ax_0 + by_0 = c
              \end{align*}

              Therefore, we showed that $x=x_0+\frac{b}{d}t, y=y_0-\frac{a}{d}t$ is a
              solution of $ax+by=c, \forall t \in \mathbb{Z}$.
        \item Now we need to show that every integer solution can be expressed in this form.

              Subtract $ax_0+by_0=ax+by=c$ togeather: \[
                  a(x-x_0) + b(y-y_0) = 0 \implies \frac{a}{d}(x-x_0) = - \frac{b}{d}(y-y_0)
              \]
              Note that $\gcd(\frac{a}{d}, \frac{b}{d}) = 1$. Therefore, $\frac{a}{d} |
                  (y-y_0)$ by the Key Lemma.

              Then let $y-y_0 = -\frac{a}{d}t$ for some $t \in \mathbb{Z}$. Substitute it
              into the equation: \[
                  \frac{a}{d}(x-x_0) = \frac{b}{d}\frac{a}{d}t
              \]
              Here we have two cases:\begin{itemize}
                  \item If $a=0$, then $by=c$. $x$ can be arbitrary and $y = \frac{c}{b}$. Then
                        $a(x_0+\frac{b}{d}t) + b(y_0-\frac{a}{d}t) = by_o = ax+by = c,\forall
                            t\in\mathbb{R}$. Note that $b\neq 0$ in this case because $(a,b)\neq(0,0)$.
                        $a\neq 0 \land b=0$ case is similar.
                  \item Now we talk about the general case. Divide both sides by $\frac{a}{d}$: \[
                            x-x_0 = \frac{b}{d}t
                        \]
                        Then $x - x_0 = \frac{b}{d}t$ and we already have $y = y_0 - \frac{a}{d}t$. We
                        have shown that every integer solution can be expressed in this form.
              \end{itemize}
    \end{itemize}
\end{proof}
\subsection{Fundedamental theorem of arithmetic}
\begin{theorem}
    $\forall n \in \mathbb{N}, n \ge 1$, $n$ can be written as a product of primes in a unique way.
\end{theorem}
Note that one prime can appear multiple times in the product. For example, $12 = 2 \cdot 2 \cdot 3$. Also, we assume that $n=1$ is an empty product which equals 1.

Usually, we collect the same primes together and write as \[
    n=p_1^{e_1}p_2^{e_2}\dots p_k^{e_k}
\]
Now $p_1, p_2, \dots, p_k$ are all distinct primes that appear in the product
and $e_1, e_2, \dots, e_k$ are the powers of the primes $\in \mathbb{N}$.

We have already proved the existence of the prime factorization in the previous
section using the Well-Ordering Principle. Now we need to prove the uniqueness.

\subsubsection{Corollary}
Let $p$ be a prime, $a_1, a_2, \dots, a_n \in \mathbb{Z}$. Then $p |
    a_1a_2\dots a_n \implies p | a_i$ for some $i$. \begin{proof}
    We prove by induction on $n$.

    \textbf{Base Case:} $n=1$. Then $p | a_1$. Obviously true.

    \textbf{Inductive Step:} Suppose the statement is true for $n=k$. We want to show that the statement is true for $n=k+1$.

    Suppose $p | a_1a_2\dots a_{k+1}$. Then $p | a_1a_2\dots a_k$ or $p | a_{k+1}$.
    By the inductive hypothesis, $p | a_i$ for some $i$.
\end{proof}

\subsubsection{Proof of the theorem}
\begin{proof}
    Let $n = p_1, p_2, \ldots, p_r = q_1, q_2, \ldots, q_s$ be two prime factorizations of $n$. We want to show that $r = s$ and $p_i = q_i$ for all $i$.

    Cancel both primes on both sides if the same primes appear in both
    factorizations. Then we have new primes on both sides that do not equal each
    other\[
        p_1p_2\dots p_a = q_1q_2\dots q_b
    \]

    Our goal now is to show that there is nothing left on both sides.

    Now by the corollary, $p_1 | q_1q_2\dots q_b$. Then $p_1 | q_i$ for some $i$.
    But $q_i$ is a prime. Therefore, $p_1 = q_i$. Then we can cancel $p_1$ and
    $q_i$ from both sides. So this is a contradiction. Therefore, $r = s$ and $p_i
        = q_i$ for all $i$. We have proved the uniqueness.
\end{proof}

Remark: Uniqueness of prime factorization fails in many number systems other
than the integers. For example: ``even integers'': In this system, then 6, 10,
30 and 50 are primes in this system. Then 300 = 6 $\cdot$ 50 = 10 $\cdot$ 30.
So the prime factorization is not unique in this system.

\subsubsection{Applications}
Proposition: Suppose $n=p_1^{d_1}\times \cdots \times p_k^{d_k}, m =
    p_1^{e_1}\times \cdots \times p_k^{e_k}$ where $p_1,\ldots,p_k$ are distinct
primes and $d_1,\ldots,d_k,e_1,\ldots,e_k \in \mathbb{N}\cup 0$. Then
$\gcd(n,m) = p_1^{\min(d_1,e_1)}\times \cdots \times p_k^{\min(d_k,e_k)}$ and
the largest common multiple $lcm(n,m) = p_1^{\max(d_1,e_1)}\times \cdots \times
    p_k^{\max(d_k,e_k)}$.

For example: \begin{align*}
    35           & =5\times7                            & = 2^0\times 5^1\times 7^1\times 11^0 \\
    154          & = 2\times7\times11                   & = 2^1\times 5^0\times 7^1\times 11^1 \\
    \gcd(35,154) & = 2^0\times 5^0\times 7^1\times 11^0 & = 7
\end{align*}

In general, prime factorization is difficult for large numbers. The Euclidean
algorithm is a much faster way to find the gcd of two numbers.

To show the Proposition is true, we need the following corollary:

Let us write the highest power of p appearing in the prime factorization of n
as $v_p(n)$. For example: $v_2(12) = 2, v_3(12) = 1, v_5(12) = 0$. Then we have
the following corollary:

If $m$ and $n$ are positive integers, then $m | n \iff v_p(m) \leq v_p(n)$ for
all primes p.\begin{proof}
    Prove both directions:
    \begin{itemize}
        \item $\implies$: Suppose $m | n$. Then $n = mk$ for some $k \in \mathbb{Z}$. Then $v_p(n) = v_p(mk) = v_p(m) + v_p(k) \ge v_p(m)$.
        \item $\impliedby$: Suppose $0\leq v_p(m) \leq v_p(n)$ for all primes p. Then $n = p_1^{v_{p_1}(n)}\times \cdots \times p_k^{v_{p_k}(n)} = m(p_1^{v_{p_1}(n)-v_{p_1}(m)}\times \cdots \times p_k^{v_{p_k}(n)-v_{p_k}(m)})$. Since $0\leq v_p(m) \leq v_p(n)$, we make sure that $\forall i, v_{p_i}(n)-v_{p_i}(m) \in \mathbb{N}\cup 0$. Therefore, $m | n$.
    \end{itemize}
\end{proof}

Now back to the Proposition: \begin{proof}
    By the Corollary above: $\forall$ common divisor $m$ and $n$ is of the form $p_1^{f_1}\ldots p_k^{f_k}$ where $f_i \leq \min(d_i,e_i)$.

    Therefore, The greatest common divisor is exactly where $f_i = \min(d_i,e_i)$.
    Therefore, $\gcd(n,m) = p_1^{\min(d_1,e_1)}\ldots p_k^{\min(d_k,e_k)}$.

    If we count for negatives, just add $\pm$ in front of the primes.

    Proof of lcm is similar. $\forall$ lcm of m and n is of the form
    $p_1^{f_1}\ldots p_k^{f_k}$ where $f_i \geq \max(d_i,e_i)$. Therefore,
    $\text{lcm}(n,m) = p_1^{\max(d_1,e_1)}\ldots p_k^{\max(d_k,e_k)}$.
\end{proof}

Corollary: $\gcd(n,m) \times \text{lcm}(n,m) = nm$. \begin{proof}
    $\max(d_i,e_i)+\min(d_i,e_i) = d_i+e_i$. Therefore, $\gcd(n,m) \times \text{lcm}(n,m) = nm$.
\end{proof}


\section{Congruences}
\begin{definition}
    Let $a, b, n \in \mathbb{Z}, n \ge 2$. We say that $a$ is \textbf{congruent} to $b$ modulo $n$ if $n | (a-b)$. We write $a \equiv b \pmod{n}$.
\end{definition}
Note that $\forall a ,n \in \mathbb{Z}, n \ge 2, a\equiv b \pmod{n}$ for some $b \in \mathbb{Z}, 0\le b \le n-1$.

Congruences mod n may be thought of as a way of focusing on the remainder and
casting out (ignoring) the quotient (multiples of n).

\subsection{Congruence Class}
\begin{definition}
    Let $n \in \mathbb{Z}, n \ge 2$. The \textbf{congruence class} of $a$ modulo $n$ is the set of all integers that are congruent to $a$ modulo $n$. We write $[a]_n$.
\end{definition}
For example: \begin{align*}
    [0]_5 & = \{\ldots, -10, -5, 0, 5, 10, \ldots\} \\
    [1]_5 & = \{\ldots, -9, -4, 1, 6, 11, \ldots\}  \\
    [2]_5 & = \{\ldots, -8, -3, 2, 7, 12, \ldots\}  \\
    [3]_5 & = \{\ldots, -7, -2, 3, 8, 13, \ldots\}  \\
    [4]_5 & = \{\ldots, -6, -1, 4, 9, 14, \ldots\}
\end{align*}
Note that $[a]_n = [b]_n \iff a \equiv b \pmod{n}$.
\subsection{Properties of congruences}
\begin{align*}
    a & \equiv a \pmod{n} \forall a, n \in \mathbb{Z}                                \\
    a & \equiv b \pmod{n} \implies b \equiv a \pmod{n}                               \\
    a & \equiv b \pmod{n} \land b \equiv c \pmod{n} \implies a \equiv c \pmod{n}     \\
    a & \equiv b \pmod{n} \land c \equiv d \pmod{n} \implies a+c \equiv b+d \pmod{n} \\
    a & \equiv b \pmod{n} \land c \equiv d \pmod{n} \implies ac \equiv bd \pmod{n}   \\
    a & \equiv b \pmod{n} \implies ac \equiv bc \pmod{n}                             \\
    a & \equiv b \pmod{n} \implies a^k \equiv b^k \pmod{n} \forall k \in \mathbb{N}
\end{align*}
Note that it is reflective, symmetric and transitive. It is also closed under addition, multiplication and exponentiation.
\subsubsection{Arithmetics on congruence classes}
Arithmetics between congruence classes: \begin{align*}
    [a]_n + [b]_n      & = [a+b]_n        \\
    [a]_n \times [b]_n & = [a \times b]_n \\
    [a]_n^k            & = [a^k]_n
\end{align*}
If $x_1 \equiv x_2 \pmod n, y_1 \equiv y_2 \pmod n$, then \begin{itemize}
    \item $x_1 + y_1 \equiv x_2 + y_2 \pmod n$
    \item $x_1y_1 \equiv x_2y_2 \pmod n$
    \item $x_1 - y_1 \equiv x_2 - y_2 \pmod n$
\end{itemize}
\begin{proof}
    Suppose $x_1-x_2=an,y_1-y_2=bn$ for some $a,b \in \mathbb{Z}$. Then
    \begin{itemize}
        \item $(x_1+y_1)-(x_2+y_2)=an+bn=(a+b)n\equiv 0 \pmod n$ Thus $x_1+y_1\equiv x_2+y_2 \pmod n$
        \item Subtraction is similar to addition proof
        \item $x_1=x_2+an,y_1=y_2+bn$ Then $x_1y_1=x_2y_2+n(a y_2+b x_2+ab)\equiv x_2y_2 \pmod n$
    \end{itemize}
\end{proof}
Using the rules above, we can calculate some insanely big numbers: \begin{align*}
    48^{100}+15\times 70002 \pmod 7 & \equiv (-1)^{100}+1\times 2 \pmod 7 \\
                                    & \equiv 1+2 \pmod 7                  \\
                                    & \equiv 3 \pmod 7
\end{align*}
Note that if $d\equiv e \pmod n$, does not imply $a^d\equiv a^e \pmod n$. For example, $2\equiv 9 \pmod 7$, but $2^2\not\equiv 2^9 \pmod 7$.
\subsubsection{Modular exponentiation}
Modular exponentiation like raising an integer to a high power modulo n, is
used in cryptography.

For large $x,y\equiv a^x \pmod n$ is easy to compute. On the other hand, given
$y$, recovering $x$ is hard. Without additional information, it is a ``Discrete
Logarithm Problem''.

A quick algorithm for computing $a^x \pmod n$ is called ``the method of
repeated squaring''. For example, $a^{13} \pmod n$ can be computed as follows:

For example: find $7^51\pmod 17$
\begin{align*}
    7^1    & \equiv 7 \pmod {17}                                                                                               \\
    7^2    & \equiv 7\times 7 \equiv 49 \equiv 15 \equiv -2 \pmod {17}                                                         \\
    7^4    & \equiv (-2)^2 \equiv 4 \pmod {17}                                                                                 \\
    7^8    & \equiv 4^2 \equiv 16 \equiv -1 \pmod {17}                                                                         \\
    7^{16} & \equiv (-1)^2 \equiv 1 \pmod {17}                                                                                 \\
    7^{32} & \equiv 1^2 \equiv 1 \pmod {17}                                                                                    \\
    7^{51} & \equiv 7^{32}\times 7^{16}\times 7^2\times 7^1 \equiv 1\times 1\times (-2)\times 7 \equiv -14 \equiv 3 \pmod {17}
\end{align*}

\subsubsection{Representations of integers}
\begin{theorem}
    Let $b\in \mathbb{Z},b\ge 2$. Then $\forall m \in \mathbb{N}$ can be written as \[
        m=a_k b^k+a_{k-1}b^{k-1}+\cdots+a_1b+a_0
    \]

    where $k\in \mathbb{N},a_0,a_1,\ldots,a_k\in \mathbb{Z},0\le a_i\le b-1$ for
    all $i$.

    Moreover, this representation is unique.
\end{theorem}
The theorem gives a representation of $m$ in base b. $a_k$ is the highest digit and $a_0$ is the lowest digit. If $b=10$, then it is the normal decimal representation. We just simply leave out the subscript $b=10$.

For example, $51=32+16+2+1=1\times 2^5+1\times 2^4+0\times 2^3+0\times
    2^2+1\times 2^1+1\times 2^0=110011_2$.

\begin{proof}
    Prove by strong induction on $n$.

    \textbf{Base Case:} $b^0\le n<b^1$. Then $n=(n)_b$.

    \textbf{Inductive Step:} Assume $n\ge b$ and the existence part of the theorem holds for any integer between $1$ and $n-1$. We want to show that the existence part of the theorem holds for $n$.

    Divide $n$ by $b$: $n=bq+r$ where $0\le r<b$. Then $0\le q<n$. By the inductive
    hypothesis, $q$ can be written as \[
        q=c_s b^s+c_{s-1}b^{s-1}+\cdots+c_1b+c_0
    \]
    Now $n=qb+r=c_s b^{s+1}+c_{s-1}b^s+\cdots+c_1b^2+c_0b+r$. Therefore, $n$ can be
    written as \[
        n=a_k b^k+a_{k-1}b^{k-1}+\cdots+a_1b+a_0
    \]
    This is the representation of $n$ in base $b$.

    For uniqueness, again by induction:

    \textbf{Base Case:} $n=b^0\le n<b^1$. Then $n=(n)_b$. It only has 1 digit and it has to be n, thus unique.

    \textbf{Inductive Step:} Assume $n\ge b$ and the uniqueness part of the theorem holds for any integer between $1$ and $n-1$. We want to show that the uniqueness part of the theorem holds for $n$.

    assume $n$ can be expressed in two ways: \begin{align*}
        n & = a_k b^k+a_{k-1}b^{k-1}+\cdots+a_1b+a_0 \\
        n & = c_s b^s+c_{s-1}b^{s-1}+\cdots+c_1b+c_0
    \end{align*}
    Then \begin{align*}
        a_0 = n \pmod b = c_0                           & \land  0\le a_0, c_0 \le b-1                      \\
        \frac{1}{b}(n-a_0)=a_k b^{k-1}+\cdots+a_2 b+a_1 & = \frac{1}{b}(n-c_0)=c_s b^{s-1}+\cdots+c_2 b+c_1
    \end{align*}
    By induction assumption, $k-1=m-1 \land a_i=c_i$ for all $i$. Therefore, $k=s \land a_i=c_i$ for all $i$.
\end{proof}
\subsubsection{Exponentiation arithmetic}
Back to repeated squaring, the goal is to compute $a^x \pmod n$.

Pre-compute: $a,a^2,a^4,a^8,\ldots,a^{2^k} \pmod n$

Now write $x$ in base 2: $x=2^{k_1}+2^{k_2}+\cdots+2^{k_m}$ where
$k_1>k_2>\cdots>k_m\ge 0$.

Then $a^x=a^{2^{k_1}}\times a^{2^{k_2}}\times \cdots \times a^{2^{k_m}} \pmod
    n$.

For example: Compute the last two digits of $53^{29}$. Equivalently, it is
$53^{29} \pmod {100}$.

Start with 29 in base 2: $29=16+8+4+1$. Then $53^{29}=53^{16}\times 53^8\times
    53^4\times 53^1 \pmod {100}$.

\begin{align*}
    53^1    & = 53 \pmod {100}                                      \\
    53^2    & = 2809 \equiv 9 \pmod {100}                           \\
    53^4    & = (53^2)^2 \equiv 81 \pmod {100}                      \\
    53^8    & \equiv 81^2 = 6561 \equiv 61 \pmod {100}              \\
    53^{16} & \equiv 61^2 = 3721 \equiv 21 \pmod {100}              \\
    53^{29} & = 21\times 61\times 81\times 53 \equiv 13 \pmod {100}
\end{align*}
\subsection{Linear congruences}
$ax\equiv b \pmod n$ is a linear congruence. We want to find all solutions of $x$ given $a,b,n$.

The case where $b=1$ is of particular interest. Here $x$ is called the inverse
of $a$ modulo $n$. We write $x=a^{-1} \pmod n$.

Note that $ax \equiv b \pmod n \iff a x = b + n k$ for some $k \in \mathbb{Z}$.
For example, $ax-nk=b$ for some $k\in \mathbb{Z}$. We know that
$ax-nk=\gcd(a,n)$ by Bezout's theorem. Therefore, $ax-nk=\gcd(a,n)$ for some
$k\in \mathbb{Z}$.

If it exists, we are interested in how many there are.
\subsubsection{Solving linear congruences}
Two solutions $x_1$ and $x_2$ are considered the same $\iff \exists y \in
    \mathbb{Z}$ s.t $ax-b=ny$ or Equivalently $ax-ny=b$.

In particular, if $\gcd(a,n)$ does not divide $b$, then the congruence
$ax\equiv b \pmod n$ has no integer solution.

\begin{theorem}
    If $d=\gcd(a,n)|b$, then the congurence $ax\equiv b \pmod n$ has exactly $d$ solutions
\end{theorem}
\begin{proof}
    Assume that $d=\gcd(a,n)|b$. Then the linear Diophantine equation $ax-b=ny \iff ax-ny=b$ for some $x,y\in \mathbb{Z}$ has a particular solution $(x_0,y_0)$ by Bezout's theorem.

    General solution: $x=x_0+\frac{n}{d}t, y=y_0-\frac{a}{d}t$ for some $t\in
        \mathbb{Z}$.

    We are only interested in $x$ here and up to multiples of $n$. In other words,
    $x_0+\frac{n}{d}t_1$ and $x_0+\frac{n}{d}t_2$ are considered the same if and
    only if $\frac{n}{d}t_1\equiv \frac{n}{d}t_2 \pmod {n}$ or equivalently,
    $\frac{n}{d}(t_1-t_2)\equiv 0 \pmod {n}$. (So that the congruence is the same)

    $\frac{n}{d}(t_1-t_2)\equiv 0 \pmod {n}\implies \frac{n}{d}(t_1-t_2)=ns \implies t_1-t_2=ds$ for some $s\in \mathbb{Z}$. Therefore, $t_1-t_2\equiv 0 \pmod d$. Thus we get exactly $d$ solutions by letting $t$ range from $0$ to $d-1$.
\end{proof}

Conclusion: Every solution is $ax\equiv b \pmod n$ is of the form
$x=x_0+\frac{n}{d}t$ for some $t\in \mathbb{Z}$ where $d=\gcd(a,n)$ and $x_0$
is a particular solution. Two solutions $x_1 = x_0+\frac{n}{d}t_1$ and $x_2 =
    x_0+\frac{n}{d}t_2$ are considered the same if and only if
$\frac{n}{d}(t_1-t_2)\equiv 0 \pmod {n} \iff t_1-t_2 \equiv 0 \pmod d$. Thus we
get exactly $d$ solutions by letting $t$ range from $0$ to $d-1$.
\subsubsection{Examples}
For example, solve $10x\equiv 3 \pmod {12}$. Here $d = \gcd(10,12) = 2$. Since
$2\nmid 3$, there are no solutions. The gcd needs to divide the congruence
value.

Another example, $10x\equiv 4 \pmod {12}$. Here we expect 2 solutions. To find
them, we start by finding a particular solution for $10x -12y = 4$. Since the
number is small, we can just guess the solution:

$x_0=4, y_0=3$ is the particular solution. Then the general solution is $x=4+\frac{12}{2}t=4+6t$ for some $t\in \mathbb{Z}$.
\begin{itemize}
    \item $t$ ranges from $0$ to $1$.
    \item When $t=0$, $x=4$.
    \item When $t=1$, $x=10$.
\end{itemize}
So the solutions are $x\equiv 4 \pmod {12}$ and $x\equiv 10 \pmod {12}$.
\subsubsection{Multuplicative inverses}
Particular interesting case when $b=1$. Then $ax\equiv 1 \pmod n$ is called the
inverse of $a$ modulo $n$. We write $x=a^{-1} \pmod n$.

In this case, the congruence has a solution if and only if $\gcd(a,n)=1$, which
also means that the congruence has exactly one solution.

Not all congruence classes have an inverse. For example, $ax\equiv 1 \pmod
    {10}$: \begin{align*}
    1\times 1 & \equiv 1 \pmod {10} \\
    3\times 7 & \equiv 1 \pmod {10} \\
    7\times 3 & \equiv 1 \pmod {10} \\
    9\times 9 & \equiv 1 \pmod {10}
\end{align*}
And the rest does not have one. It is because the rest have a gcd with 10 that is greater than 1.

Corollary: Let $p$ be a prime, then every $a\not\equiv 0 \pmod p$ has an
inverse modulo $p$. 
\begin{proof}
    It is because $\gcd(a,p)=1$. By Bezout's theorem, $ax-py=1$ for some $x,y\in \mathbb{Z}$. Therefore, $ax\equiv 1 \pmod p$. Therefore, $x$ is the inverse of $a$ modulo $p$.
\end{proof}


Corollary: Let $p$ be a prime, then $a\equiv a^{-1} \pmod p \iff a \equiv \pm 1
    \pmod p$.
\begin{proof}
    Indeed multiplying both sides by $a$, we get $a\equiv a^{-1} \pmod p \implies a^2\equiv 1 \pmod p \iff (a-1)(a+1)\equiv 0 \pmod p \iff a\equiv \pm 1 \pmod p$.

    If $a\not \equiv 1$ then $a-1$ has a multiplicative inverse modulo $p$. Denote
    this multiplicative inverse by $b$. Time both sides by $b$ Then
    $b(a-1)(a+1)\equiv 0 \pmod p \implies a+1\equiv 0 \pmod p \implies a\equiv -1
        \pmod p$. Therefore, $a\equiv b \pmod p$.

    Similarly we can find that $a\not \equiv -1 \implies a\equiv 1 \pmod p$. So we
    proved the corollary.
\end{proof}

For example, solve $17x\equiv 1 \pmod {55}$. First, find the gcd of 17 and 55: \begin{align*}
    55 & = 3\times 17+4 \\
    17 & = 4\times 4+1  \\
    4  & = 4\times 1
\end{align*}
Now use back substitution: \begin{align*}
    1 & = 17-4\times 4               \\
      & = 17-4\times (55-3\times 17) \\
      & = 13\times 17-4\times 55
\end{align*}
Therefore, $13\times 17\equiv 1 \pmod {55}$. So the solution is $x\equiv 13 \pmod {55}$.
\subsection{Chinese Remainder Theorem}
\begin{theorem}
    Suppose $n_1,n_2 \in \mathbb{N}$ and $\gcd(n_1,n_2)=1$ (Relatively prime). Then for any integers $a_1, a_2, \exists x\in \mathbb{Z}$ s.t \begin{align*}
        x & \equiv a_1 \pmod {n_1} \\
        x & \equiv a_2 \pmod {n_2}
    \end{align*}
    Moreover, the solution is unique modulo $n_1n_2$.
\end{theorem}
Here uniqueness means the following: Suppose \begin{align*}
    x & \equiv a_1 \pmod {n_1} &  & y \equiv a_1 \pmod {n_1} \\
    x & \equiv a_2 \pmod {n_2} &  & y \equiv a_2 \pmod {n_2}
\end{align*}
Then $x\equiv y \pmod {n_1n_2}$.
\subsubsection{Proof of uniqueness}
\begin{proof}
    Assume $x\equiv y \equiv a_1 \pmod {n_1}$ and $x\equiv y \equiv a_2 \pmod {n_2}$. Then $n_1 | (x-y)$ and $n_2 | (x-y)$.

    Then $x-y=0\pmod {n_1}$ and $x-y=0\pmod {n_2}$. Then $n_1 | (x-y)$ and $n_2 |
        (x-y)$.

    Note that any common multiple of $n_1,n_2$ is divisible by \[
        \text{lcm}(n_1,n_2)=\frac{n_1 n_2}{\gcd(n_1,n_2)}=n_1n_2.
    \]

    Then $n_1n_2 | (x-y)$. $(x-y)$ is a common divisor.

    Therefore, $x\equiv y \pmod {n_1n_2}$.
\end{proof}
Here we can also apply the \hyperref[sec:key-lemma]{Key Lemma} to prove the existence of the solution.

\subsubsection{Proof of existence}
\begin{proof}
    Let $n=n_1 n_2$. To each congruence class $[x] \pmod n$, associate a pair of integers $(z_1,z_2)$ where $z_1\equiv x \pmod {n_1}$ and $z_2\equiv x \pmod {n_2}$. Pick $0\le z_1 < n_1$ and $0\le z_2 < n_2$.

    Note that there are exactly $n_1 n_2=n$ such pairs. And there are exactly $n$
    congruence classes $[x] \pmod n$.

    Moreover, by uniqueness, we never associate the same pair $z_1, z_2$ to two
    different congruence classes $[x] \pmod n$.

    By the Pigeonhole Principle, we conclude that every pair $(a_1,a_2)$ is
    associated with some congruence class $[x] \pmod n$. Therefore, the solution
    exists.
\end{proof}
Example illustrating the theorem: $n_1=3,n_2=5$ \begin{align*}
    x & \equiv 2 \pmod 3 \\
    x & \equiv 3 \pmod 5
\end{align*}
Where $0\le x\le 14$. Can list out all the combinations to find it, but we need a better algorithm for larger numbers.
\subsubsection{Algorithm}
Note that we only need to solve two systems of linear congruences: \begin{align*}
    y & \equiv 1 \pmod {n_1} &  & z \equiv 0 \pmod {n_1} \\
    y & \equiv 0 \pmod {n_2} &  & z \equiv 1 \pmod {n_2}
\end{align*}
because we can do $x=a_1 y+a_2 z$ to get $x\equiv a_1 \pmod {n_1}$ and $x\equiv a_2 \pmod {n_2}$. by solving:
\begin{align*}
    x\equiv a_1y+a_2 z \equiv a_1 \times 1 + a_2 \times 0 \pmod {n_1} \\
    x\equiv a_1y+a_2 z \equiv a_1 \times 0 + a_2 \times 1 \pmod {n_2}
\end{align*}


Now the problem comes to how do we solve \begin{align*}
    y & \equiv 1 \pmod {n_1} \\
    y & \equiv 0 \pmod {n_2}
\end{align*}

Set $y=n_2 t$ into the first congruence. $n_2 t \equiv 1 \pmod {n_1}$. Since
$\gcd(n_1,n_2)=1$, we can find $t=t_1$ such that $n_2 t_1 \equiv 1 \pmod
    {n_1}$. Then $t_1$ is the multiplicative inverse of $n_2$ modulo $n_1$. Then
$y=n_2 t_1$ is the solution to the first congruence.

Similarly, we can find $z=n_1 t_2$ is the solution to the second congruence.

In Summary, the solution to the system of linear congruences \begin{align*}
    x & \equiv a_1 \pmod {n_1} \\
    x & \equiv a_2 \pmod {n_2}
\end{align*}
is \[
    x=a_1 n_2 t_1+a_2 n_1 t_2
\]
where $t_1$ is the multiplicative inverse of $n_2$ modulo $n_1$ and $t_2$ is
the multiplicative inverse of $n_1$ modulo $n_2$.

So basically: $t_1\equiv n_2^{-1} \pmod {n_1}$ and $t_2\equiv n_1^{-1} \pmod {n_2}$.
\subsubsection{Example}
For example, solve \begin{align*}
    x & \equiv 2 \pmod 3 \\
    x & \equiv 3 \pmod 7
\end{align*}
Here: $n_1 = 7, n_2 = 3, a_1 = 3, a_2 = 2$. 

First, find the multiplicative inverse of 3 modulo 7. \begin{align*}
    7 & = 2\times 3+1 \\
    1 & = 7-2\times 3
\end{align*}
Therefore, $3^{-1}\equiv -2 \equiv 5 \pmod 7$.

Then find the multiplicative inverse of 7 modulo 3. \begin{align*}
    3 & = 1\times 3+0
\end{align*}
Therefore, $7^{-1}\equiv 1 \pmod 3$.

Then the solution is \[
    x\equiv a_1 t_1 n_2+a_2 t_2 n_1\equiv 3\times 5\times 3+2\times 1\times 7\equiv 45+14 \equiv 17 \pmod {21}
\]

Another example, revisit: Find $17^{-1}\pmod {55}$.

Use the fact that $55=11\times 5, \gcd(11,5)=1$. Then we can solve: \begin{align*}
    17 x & \equiv 1 \pmod {11} & \implies 6x\equiv 1 \pmod {11} \\
    17 x & \equiv 1 \pmod {5} & \implies 2x\equiv 1 \pmod {5}
\end{align*}

Easily find that $x\equiv 2 \pmod {11}$ and $x\equiv 3 \pmod {5}$. Then we need to recover $x$. Know that: \[
    x\equiv a_1 t_1 n_2 + a_2 t_2 n_1 \pmod {n_1n_2}
\]
And we know that $n_1 = 5, n_2 = 11, a_1 = 3, a_2 = 2$. Then we can find that \begin{align*}
    t_1 & \equiv n_2^{-1} & \equiv 11^{-1} & \equiv 1 \pmod 5 \\
    t_2 & \equiv n_1^{-1} & \equiv 5^{-1}  & \equiv -2 \pmod {11}
\end{align*}
Then $x \equiv 3\times 1\times 11+2\times (-2)\times 5 \equiv 33+90 \equiv 123 \equiv 13 \pmod {55}$.
\subsection{More general version of CRT}
\begin{theorem}
    The system of $r$ congruences: \begin{align*}
        x & \equiv a_1 \pmod {n_1} \\
        x & \equiv a_2 \pmod {n_2} \\
          & \vdots                 \\
        x & \equiv a_r \pmod {n_r}
    \end{align*}
    has a unique solution modulo $N=n_1n_2\ldots n_r$ if $n_1,n_2,\ldots,n_r$ are pairwise relatively prime.
\end{theorem}
\subsubsection{proof}
\begin{proof}
    Induction on $r$. The base case $r =1 $ is clear and $r=2$ is the Chinese Remainder Theorem, discussed above.

    Inductive step: Assume that the theorem holds for $r-1$ numbers. We want to show that the theorem holds for $r$.
    
    Let $b$ be the unique solution to the system of \begin{align*}
        x &\equiv a_{r-1} \pmod {n_{r-1}} \\
        x &\equiv a_r \pmod {n_r}
    \end{align*}

    Then we can reduce the system of $r$ congruences to a system of $r-1$ congruences: \begin{align*}
        x &\equiv a_1 \pmod {n_1} \\
        x &\equiv a_2 \pmod {n_2} \\
          &\vdots                 \\
        x &\equiv a_{r-2} \pmod {n_{r-2}} \\
        x &\equiv b \pmod {n_{r-1}n_r}
    \end{align*}
    Note that there are only $r-1$ congruences now and all $n_i$ are relatively prime to each other. 
    
    By induction assumption, the system has a unique solution modulo $N=n_1n_2\ldots n_{r-2}n_{r-1}n_r$. Therefore, the system of $r$ congruences has a unique solution modulo $N=n_1n_2\ldots n_r$.
\end{proof}
\subsubsection{Algorithm}
In practice, to solve the system of linear congruences, we use the recursive step as shown in the proof. 

Alternatively, there is a formula that allows us to find x in one step.

Let $N = n_1n_2\ldots n_r$. $N_i = N/n_i$. Then the solution to the system of linear congruences is \[
    x = \sum_{i=1}^r a_i N_i t_i
\]
where $t_i$ is the multiplicative inverse of $N_i$ modulo $n_i$.

To see that $x\equiv a_i \pmod {n_i}$, note that $N_i t_i \equiv 1 \pmod {n_i}$ and $N_i t_i \equiv 0 \pmod {n_j}$ for all $j\ne i$.

For example, solve \begin{align*}
    x & \equiv 1 \pmod 2 \\
    x & \equiv 2 \pmod 3 \\
    x & \equiv 3 \pmod 5 \\
    x & \equiv 4 \pmod 7 \\
    x & \equiv 5 \pmod {11}
\end{align*}
\begin{align*}
    N & = 2\times 3\times 5\times 7\times 11 = 2310 \\
    N_1 & = 2310/2 = 1155 \\
    t_1 & \equiv (3\times 5 \times 7 \times 11)^{-1} \equiv (1^4)^{-1} \equiv 1 \pmod 2 \\
    N_2 & = 2310/3 = 770 \\ 
    t_2 & \equiv (2\times 5 \times 7 \times 11)^{-1} \equiv ((-1)\times (-1)\times 1 \times (-1))^{-1} \equiv 2 \pmod 3 \\
    N_3 & = 2310/5 = 462 \\ 
    t_3 & \equiv (2\times 3 \times 7 \times 11)^{-1} \equiv (2\times 3 \times 2 \times 1)^{-1} \equiv 3 \pmod 5 \\
    N_4 & = 2310/7 = 330 \\
    t_4 & \equiv (2\times 3 \times 5 \times 11)^{-1} \equiv (2\times 3 \times (-2) \times 4)^{-1} \equiv 1 \pmod 7 \\
    N_5 & = 2310/11 = 210 \\
    t_5 & \equiv (2\times 3 \times 5 \times 7)^{-1} \equiv (2\times 3 \times 5 \times 7)^{-1} \equiv (10\times 21)^{-1} \equiv 1 \pmod {11}
\end{align*}
Thus $x = 1\times 1155\times 1 + 2\times 770\times 2 + 3\times 462\times 3 + 4\times 330\times 1 + 5\times 210\times 1 \equiv 1523 \pmod {2310}$.

Another example: Find $x=8^{10003}\pmod {105}$.

Note that $105=3\times 5\times 7$. Then we can solve the system of linear congruences: \begin{align*}
    x & \equiv 8^{10003} \pmod 3 \\
    x & \equiv 8^{10003} \pmod 5 \\
    x & \equiv 8^{10003} \pmod 7
\end{align*}
then put them all together using CRT. 

\begin{align*}
    x&\equiv 8^{10003} \equiv {(-1)}^{10003} \equiv -1 \equiv 2 \pmod 3 \\
    x&\equiv 8^{10003} \equiv {(-2)}^{10003} \equiv {((-2)^2)}^{5001} \times (-2)\equiv (-1)^{5001}\times (-2) \equiv 2 \pmod 5 \\
    x&\equiv 8^{10003} \equiv 1^{10003} \equiv 1 \pmod 7
\end{align*}

3,5,7 are pairwise relatively prime. Then we can use the formula to find the solution:
\begin{align*}
    a_1 & = 2, N_1 = 35, t_1 \equiv (N_1)^{-1} \pmod 3 \\
    a_2 & = 2, N_2 = 21, t_2 \equiv (N_2)^{-1} \pmod 5 \\
    a_3 & = 1, N_3 = 15, t_3 \equiv (N_3)^{-1} \pmod 7 \\
\end{align*}
\begin{align*}
    t_1 &\equiv 35^{-1} \equiv (-1)^{-1}\equiv 2 \pmod 3 \\
    t_2 &\equiv 21^{-1} \equiv 1^{-1} \equiv 1 \pmod 5 \\
    t_3 &\equiv 15^{-1} \equiv 1^{-1} \equiv 1 \pmod 7 \\
\end{align*}

\[
    x = 2\times 35\times 2 + 2\times 21\times 1 + 1\times 15\times 1 \equiv 197 \equiv 92 \pmod {105}
\]

\subsubsection{More generalized CRT}
Now consider a system of $\gcd = d > 1$.

solve $x\equiv 2 \pmod 4, x\equiv 1 \pmod {12}$. There is no solution since the first congruence implies $x$ is even and the second congruence implies $x$ is odd.

solve $x\equiv 1 \pmod 8, x\equiv 1 \pmod {12}$. Obviously, $x=1,25$ are solutions to it. Here it has solutions, but it is not unique under modulo $8\times 12=96$.

\begin{theorem}
    The system of linear congruences \begin{align*}
        x & \equiv a_1 \pmod {n_1} \\
        x & \equiv a_2 \pmod {n_2} \\
          & \vdots                 \\
        x & \equiv a_r \pmod {n_r}
    \end{align*}
    has a solution if and only if $a_i\equiv a_j \pmod {\gcd(n_i,n_j)}$ for all $i,j$. Moreover, the solution is unique modulo $N=\text{lcm}(n_1,n_2,\ldots,n_r)$.
\end{theorem}
Here when $\gcd(n_i,n_j)=1$, the congruence is the same as the original version of the Chinese Remainder Theorem. When $\gcd(n_i,n_j)>1$, the congruence is the same as the more generalized version of CRT.

To prove it we need the following Lemma:\[
    x\equiv y \pmod n \iff ax \equiv ay \pmod {an}
\]
\begin{proof}
    Let $z=x-y$. Then the lemma can be rewritten as \[
        z\equiv 0 \pmod n \iff az \equiv 0 \pmod {an}
    \]
    
    Forward direction: $\exists w\in \mathbb{Z}$ s.t $z=wn \implies \exists w\in \mathbb{Z}$ s.t $az=awn$.

    Backward direction: $\exists w\in \mathbb{Z}$ s.t $az=awn \implies \exists w\in \mathbb{Z}$ s.t $z=wn$.

    Therefore, the lemma is proved.
\end{proof}

Now back to the proof of the theorem.
\begin{proof}
    If there is a solution $x$ to the linear system, then reducing to modulo $d$, we obtain \begin{align*}
        x & \equiv a_1 \pmod d \\
        x & \equiv a_2 \pmod d \\
          & \vdots             \\
        x & \equiv a_r \pmod d
    \end{align*}
    Then $a_i\equiv a_j \pmod d$ for all $i,j$.

    Conversely, if $a_i\equiv a_j \pmod d$ for all $i,j$, then we can find $x_i$ such that $x_i\equiv a_i \pmod {n_i}$. Then $x_i\equiv a_i \pmod d$. Then $x_i\equiv a_j \pmod d$. Then $x_i\equiv x_j \pmod d$. Then $x_i=x_j+dn$ for some $n\in \mathbb{Z}$. Then $x_i\equiv x_j \pmod {dn_i}$. Then $x_i\equiv x_j \pmod N$. Therefore, the solution is unique modulo $N$.
\end{proof}
\subsection{Divisibility criteria}
$n=(a_m a_{m-1}\ldots a_1 a_0)_b=a_m b^m+a_{m-1}b^{m-1}+\cdots+a_1b+a_0$.

\subsubsection{Base 10 divisibility criteria}
The idea is to find a pattern in the congruence of each power of $10$ modulo $n$.

For example:\begin{itemize}
    \item $10^k\equiv 1 \pmod 9, \forall k\ge 0$. Note that it is true for modulo 3 as well.\begin{itemize}
        \item Consider $n=114000$ Then $1+1+4+0+0+0=6$ and $6\equiv 0 \pmod 3$, but $6\not\equiv 0 \pmod 9$. Therefore, $n\not\equiv 0 \pmod 9$.
    \end{itemize}
    \item For divisibility by 11, $10^0 \equiv 1, 10^1\equiv -1, 10^2\equiv 1, 10^3\equiv -1, \ldots$. Then $10^k\equiv (-1)^k \pmod {11}$.\begin{itemize}
        \item Consider $n=123456$. Then $6-5+4-3+2-1=3$, not congruent to 0 mod 11. So n is not divisible by 11.
    \end{itemize}
\end{itemize}
In particular, $(a_m a_{m-1}\ldots a_1 a_0)$ is divisible by $2^k$ if and only if $(a_k a_{k-1}\ldots a_1 a_0)$ is divisible by $2^k$. Similarly, $(a_m a_{m-1}\ldots a_1 a_0)$ is divisible by $5^k$ if and only if $(a_k a_{k-1}\ldots a_1 a_0)$ is divisible by $5^k$.

This is because $10^k\equiv 0 \pmod {2^k}$ if and only if $10^k\equiv 0 \pmod {5^k}$.
For example, $n=114000$ is divisible by $2^4,5^3$ not $2^5,5^4$.\begin{itemize}
    \item So for $2^4$, $114000\equiv 4\times 10^3 \equiv 0 \pmod {2^4}$.
    \item For $2^5$, $114000\equiv 10^4+ 4 \times 10^3 \equiv 16 \pmod {2^5}$. Thus not divisible by $2^5$.
    \item For $5^3$, $114000\equiv 0 \pmod {5^3}$.
    \item For $5^4$, $114000\equiv 10^3+4\times 10^2 \equiv 400 \pmod {5^4}$. Thus not divisible by $5^4$.
\end{itemize}

For mod 11, $10^k\equiv (-1)^k \pmod {11}$. For example, $n=123456$ is not divisible by 11 because $10^6\equiv 1 \pmod {11}$ and $1-2+3-4+5-6\equiv 7 \pmod {11}$. This is because $10\equiv -1 \pmod 11$

With the same idea, divisibility by 101: Use the congruence $10^2 \equiv -1 \pmod {101}$.

\begin{align*}
    n = (a_0+10a_1)+10^2(a_2+10a_3)+\cdots+10^{2k}(a_k+10a_{k+1})\\\equiv (a_0+10a_1)-10^2(a_2+10a_3)+\cdots \pmod {101}\\
    = (a_0+10a_1)-(a_2+10a_3)+ (a_4 +10 a_5)\cdots \pmod {101}\\
    = (a_1 a_0)-(a_3 a_2)+ (a_5 a_4)\cdots \pmod {101}
\end{align*}

For example: n=123456789 is not divisible by 101 because $10^2\equiv -1 \pmod {101}$ and $89 - 67 + 45 - 23 + 1\equiv 22+22+1 \equiv 45 \pmod {101}$.

Similarly for 1001: $10^3\equiv -1 \pmod {1001}$. Now break up into 3-digit bits. $n\equiv (a_2 a_1 a_0)-(a_5 a_4 a_3)+(a_8 a_7 a_6)\cdots \pmod {1001}$. For example, $n=234934$ is not divisible by 1001 because $934-234\equiv 700 \pmod {1001}$.

Note that $1001 = 7 \times 11 \times 13$. Therefore, $n$ is divisible by 1001 if and only if $n$ is divisible by 7, 11, and 13.

\begin{itemize}
    \item Is $n=234934$ divisible by 7? $234-934\equiv 700\equiv 0 \pmod 7$.
    \item Is $n=234934$ divisible by 11? $934-234\equiv 700\equiv 7 \pmod {11}$.
    \item Is $n=234934$ divisible by 13? $934-234\equiv 700\equiv 7\times 2^5 \times 5^2 \pmod {13}$.
\end{itemize}
   Therefore, $n=234934$ is not divisible by 1001.

\section{Fermat's Theorems}
\subsection{Wilson's Theorem}
\begin{theorem}
    Let $p$ be a prime. Then $(p-1)!\equiv -1 \pmod p$.
\end{theorem}
\begin{proof}
    When $p=2$, this is obvious. 

    Assume now that $p>3\implies$ p is odd. 

    Consider the set $S=\{1,2,\ldots,p-1\}$. Then $S$ is a group under multiplication modulo $p$. Note that $\forall s \in S, \exists s^{-1} \in S$ such that $ss^{-1}\equiv 1 \pmod p$. 

    However, none of them is its own inverse except 1. It is because  $x^2\equiv 1 \pmod p \iff (x-1)(x+1)\equiv 0 \pmod p$. Then $x\equiv \pm 1 \pmod p$. A property of prime number modulo discussed in the previous section.

    Then we can pair up the elements in $S$ into pairs $(s,s^{-1})$. Then the product of each pair is congruent to 1 modulo $p$. Note that $p$ is odd, then $p-1$ is even. 

    Claim: all numbers in $S$ except 1 and the last term are paired up. 

    Note that the last term is $p-1$. Then $p-1\equiv -1 \pmod p$. Then $p-1$ is its own inverse. Only the term $\pm 1$ is its own inverse and left out.

    Therefore, $(p-1)!\equiv 1\times 2\times \cdots \times (p-2)\times (p-1) \equiv 1\times 2\times \cdots \times (p-2)\times 1 \equiv -1 \pmod p$.

     Therefore, the product of all the elements in $S$ is congruent to -1 modulo $p$.
\end{proof}
For example: $10!=1\times (2\times 6)\times (3 \times 4)\times (5\times 9)\times (7\times 8)\times 10\equiv -1 \pmod {11}$.
\subsection{Fermat's Little Theorem}
\begin{theorem}
    Let $p$ be a prime and $a$ be an integer such that $p\nmid a$. Then $a^{p-1}\equiv 1 \pmod p$.
\end{theorem}
\begin{proof}
    Claim: $a,2a,3a,\ldots,(p-1)a$ are all distinct modulo $p$.

    if $ia\equiv ja \pmod p$, then $ia-ja\equiv 0 \pmod p \implies (i-j)a\equiv 0 \pmod p$. Since $p\nmid a$, then $p\mid (i-j)$. Since $0<i,j<p$, then $i=j$.

    Thus the $p-1$ distinct congruences $a,2a,3a,\ldots,(p-1)a$ are all distinct modulo $p$. Which is equivalent to the $p-1$ distinct congruences $1,2,3,\ldots,p-1$.

    Then we can multiply all the congruences together to get \[
        a\times 2a\times 3a\times \cdots \times (p-1)a\equiv 
        a^{p-1}(p-1)!\equiv (p-1)!\pmod p
    \]
    by Wilson's Theorem, $(p-1)!\equiv -1 \pmod p$. Therefore, $a^{p-1}\equiv 1 \pmod p$.
\end{proof}
\subsubsection{Corollaries}
Corollary 1: Let $p$ be a prime and $a$ be an integer. Then $a^p\equiv a \pmod p$.
\begin{proof}
    If $a \equiv 0 \pmod p$, then $a^p\equiv 0 \equiv a \pmod p$. If $a\not\equiv 0 \pmod p$, then $a^{p-1}\equiv 1 \pmod p$. Then $a^p\equiv a \pmod p$.
\end{proof}
Corollary 2: Let $p$ be a prime and $a$ be an integer. If $d \equiv e \pmod {p-1}$, then $a^d\equiv a^e \pmod p$.

Note that Corollary 1 is a special case of Corollary 2 when $d=p,e=1 \implies a^p\equiv a^1 \pmod p$.
\begin{proof}
    May assume $d \ge e, d-e = k (p-1)$ for some $k\in \mathbb{Z},k\ge 0$. Then $a^d\equiv a^{e+k(p-1)}\equiv a^e(a^{p-1})^k\equiv a^e \pmod p$.

    We used Fermat's Little Theorem in the last step where $a^{p-1}\equiv 1 \pmod p \implies (a^{p-1})^k\equiv 1^k\equiv 1 \pmod p$.
\end{proof}
\subsubsection{Example}
Find $2^{180}\pmod {89}$.

Note that $89$ is a prime. Then $2^{88}\equiv 1 \pmod {89}$ by Fermat's Little Theorem. Then $2^{180}\equiv 2^{88\times 2+4}\equiv (2^{88})^2\times 2^4\equiv 1^2\times 16\equiv 16 \pmod {89}$.
\subsection{Euler's Theorem}
\subsubsection{Euler's phi-function}
\begin{definition}
    Let $n\in \mathbb{N}$. The Euler's phi-function $\phi(n)$ is the number of positive integers less than $n$ that are relatively prime to $n$.
\end{definition}
For example: \begin{align*}
    \phi (1) & = 1 && \{1\} \\
    \phi (2) & = 1 && \{1\} \\
    \phi (3) & = 2 && \{1,2\} \\
    \phi (4) & = 2 && \{1,3\} \\
    \phi (5) & = 4 && \{1,2,3,4\} \\
    \phi (6) & = 2 && \{1,5\} \\
    \phi (7) & = 6 && \{1,2,3,4,5,6\} \\
    \phi (8) & = 4 && \{1,3,5,7\} \\
    \phi (9) & = 6 && \{1,2,4,5,7,8\} \\
    \phi (10) & = 4 && \{1,3,7,9\} \\
    \phi (11) & = 10 && \{1,2,3,4,5,6,7,8,9,10\} 
\end{align*}
Similarly, we can conclude that $\phi(p)=p-1$ for any prime $p$.
\subsubsection{Euler's Theorem}
\begin{theorem}
    Let $n\in \mathbb{N}$ and $a\in \mathbb{Z}$ such that $\gcd(a,n)=1$. Then $a^{\phi(n)}\equiv 1 \pmod n$.
\end{theorem}
Note that Fermat's Little Theorem is a special case of Euler's Theorem when $n$ is a prime.

For example: $3^{\phi(10)}\equiv 3^4\equiv 1 \pmod {10}$. It is due to the fact that $\gcd(3,10)=1$ and $\phi(10)=4$.

\begin{lemma}
    $\gcd(a,n)=1 \land \gcd(b,n)=1 \iff \gcd(ab,n)=1$.
\end{lemma}
\begin{proof}
    $\gcd(a,n)=1 \land \gcd(b,n)=1 \implies \exists x,y\in \mathbb{Z}$ such that $ax+ny=1$ and $\exists u,v\in \mathbb{Z}$ such that $bu+nv=1$. Then $ab(xu)+n(axv+nyu+bv)=1$. Therefore, $\gcd(ab,n)=1$.

    $\gcd(ab,n)=1 \implies \exists x,y\in \mathbb{Z}$ such that $abx+ny=1$. Then $\gcd(a,n)=1$ and $\gcd(b,n)=1$.
\end{proof}
\subsubsection{Proof of Euler's Theorem}
\begin{proof}
    We use a similar proof to Fermat's Little Theorem.

    Let $S=\{a_1,a_2,\ldots,a_{\phi(n)}\}$ be the set of all positive integers less than $n$ that are relatively prime to $n$. Then $S$ is a group under multiplication modulo $n$.

    Claim: $aS=\{aa_1,aa_2,\ldots,aa_{\phi(n)}\}$ is the same set as $S$ if $a$ is relatively prime to $n$.

    If $aa_i\equiv aa_j \pmod n$, then $a_i\equiv a_j \pmod n$ because a is invertible modulo $n$. Then $a_i=a_j$ since $S$ is a set of distinct elements. Therefore, $aS=S$. $aS$ is another group under multiplication modulo $n$. The product of all elements in $aS$ should be congruent to the product of all elements in $S$.

    Then we can multiply all the elements in $S$ together to get \[
        a_1a_2\cdots a_{\phi(n)}\equiv a^{\phi(n)}(a_{1}a_{2}\cdots a_{\phi(n)})\equiv (a_{1}a_{2}\cdots a_{\phi(n)})\pmod n
    \]

    Thus $a^{\phi(n)}\equiv 1 \pmod n$.
\end{proof}

\begin{lemma}\label{lem:phi power}
    Suppose $n=p^r$ for some prime $p$ and $r\in \mathbb{N}$. Then \[\phi(n)=p^r-p^{r-1}=p^{r-1}(p-1)=n(1-\frac{1}{p})=n\frac{p-1}{p}.\]
\end{lemma}
\begin{proof}
    We know that$\phi(n)=n$ minus the number of multiples of $p$ less than $n$. 

    The number of multiples of $p$ less than $n$ is $\frac{n}{p}$ since $n=p^r$. Then $\phi(n)=n-\frac{n}{p}=n(1-\frac{1}{p})=n\frac{p-1}{p}$.
\end{proof}
\begin{lemma}\label{lem:phi morphism}
    If $\gcd(m,n)=1$, then $\phi(mn)=\phi(m)\phi(n)$.
\end{lemma}
\begin{proof}
    By the Chinese Remainder Theorem, there is a bijection between $a\pmod {mn}$ and $(a_1\pmod m,a_2\pmod n)$, where 
    
    \[\begin{cases}
        a\equiv a_1 \pmod m \\
        a\equiv a_2 \pmod n
    \end{cases}\]

    Then $\gcd(a,mn)=1 \iff \gcd(a_1,m)=1 \land \gcd(a_2,n)=1$. 
    
    There are $\phi(m)$ choices for $a_1$ and $\phi(n)$ choices for $a_2$. 
    
    Then there are $\phi(m)\phi(n)$ choices for $a$. 
    
    Then $\phi(mn)=\phi(m)\phi(n)$.
\end{proof}
\subsubsection{Find the $\phi$ function}
\begin{theorem}
    Let $n=p_1^{r_1}p_2^{r_2}\ldots p_k^{r_k}$ be the prime factorization of $n$ where $p_1,p_2,\ldots,p_k$ are distinct primes and $r_1,r_2,\ldots,r_k$ are positive integers.
    
    Then \[\phi(n)=n(1-\frac{1}{p_1})(1-\frac{1}{p_2})\ldots (1-\frac{1}{p_k})=p_1^{r_1-1}(p_1-1)p_2^{r_2-1}(p_2-1)\ldots p_k^{r_k-1}(p_k-1)\]
\end{theorem}

Basically, it is a direct application of the previous 2 lemmas. 

\begin{proof}
    Using the previous \hyperref[lem:phi morphism]{phi morphism lemma} and \hyperref[lem:phi power]{phi power lemma} recursively, we can find that \[\phi(n)=\phi(p_1^{r_1})\phi(p_2^{r_2})\ldots \phi(p_k^{r_k})=p_1^{r_1-1}(p_1-1)p_2^{r_2-1}(p_2-1)\ldots p_k^{r_k-1}(p_k-1)\]
\end{proof}

For example: $\phi(100)=\phi(2^2)\phi(5^2)=2^1(2-1)5^1(5-1)=40$.

Note that we need the prime decomposition of $n$ to find $\phi(n)$. Otherwise, it is hard to find $\phi(n)$ directly.
\subsubsection{More on $\phi$ funciton}
\begin{theorem}
    $\sum_{d|n}\phi(d)=n$. 
    
    Here the sum is taken over all positive divisors of $n$.
\end{theorem}
Note the use of the formula with the prime decomposition of $n$: \[\phi(n)=p_1^{r_1-1}(p_1-1)p_2^{r_2-1}(p_2-1)\ldots p_k^{r_k-1}(p_k-1)\]

For example: $n=6$ then $\sum_{d|6}\phi(d)=\phi(1)+\phi(2)+\phi(3)+\phi(6)=1+1+2+2=6$.

For example: $n=12$ then $\sum_{d|12}\phi(d)=\phi(1)+\phi(2)+\phi(3)+\phi(4)+\phi(6)+\phi(12)=1+1+2+2+2+4=12$.

For example: $n=81=3^4$ then $\sum_{d|81}\phi(d)=\phi(1)+\phi(3)+\phi(9)+\phi(27)+\phi(81)=1+2+6+18+54=81$.

Now try to prove it:\begin{proof}
    Subdivide the integers $[0,n-1]$ into classes $C_e$ where $C_e=\{x\in [0,n-1]:\gcd(x,n)=e\}$ for $e=1,2,\ldots,n$. Note that $|C_e|=\phi(e)$ since $x\in C_e \iff \gcd(x,n)=e$.

    More generally, the number of elements in $C_e$ is $\phi(\frac{n}{e})$. 
    
    It is because $\gcd(x,n)=e \iff e|n \land e | x \iff \gcd(\frac{x}{e},\frac{n}{e})=1$. Then the number of elements in $C_e$ is $\phi(\frac{n}{e})$.

    Since the classes $C_e$ are disjoint and their union is $[0,n-1]$, then $|C_1|+|C_2|+\cdots+|C_n|=n$. Then we conclude \[
        n=\sum_{e|n}|C_e|=\sum_{e|n}\phi(\frac{n}{2})=\sum_{d|n}\phi(d)
    \]
    Here $d=\frac{n}{e}$ ranges over the positive divisors of $n$.
\end{proof}

To illustrate the proof, consider $n=12$. Then \begin{align*}
    C_{12} & = \{0\} && \phi(\frac{12}{12})=1 \\
    C_6 & = \{6\} && \phi(\frac{12}{6})=2 \\
    C_4 & = \{4,8\} && \phi(\frac{12}{4})=2 \\
    C_3 & = \{3,9\} && \phi(\frac{12}{3})=2 \\
    C_2 & = \{2,10\} && \phi(\frac{12}{2})=2 \\
    C_1 & = \{1,5,7,11\} && \phi(\frac{12}{1})=4 \\
\end{align*}
Then $|C_{12}|+|C_6|+|C_4|+|C_3|+|C_2|+|C_1|=1+1+2+2+2+4=12$.

Note that how the sets are disjoint. 
\subsection{Multuplicative}
The Euler $\phi$ function is multiplicative. That is, if $\gcd(m,n)=1$, then $\phi(mn)=\phi(m)\phi(n)$.

Here ``arithmetic'' means defined on the set of positive integers, ``multiplicative'' means $f(mn)=f(m)f(n)$ for all $m,n$ such that $\gcd(m,n)=1$.
\begin{lemma}
    If $n=p_1^{r_1}p_2^{r_2}\ldots p_k^{r_k}$ is the prime factorization of $n$ where $p_1,p_2,\ldots,p_k$ are distinct primes and $r_1,r_2,\ldots,r_k$ are positive integers, then \[
        f(n)=\prod_{i=1}^k f(p_i^{r_i})
    \]
\end{lemma}
This can be proved by recursively applying the multiplicative property of $f(n)$.

\begin{theorem}
    If $f(n)$ is a multiplicative arithmetic function, then so is $F(n)=\sum_{d|n}f(d)$.

    Here the sum is taken over all positive divisors of $n$.
\end{theorem}
Example: If $f(n)$ is the Euler $\phi$ function, then $F(n)=\sum_{d|n}\phi(d)=n$ by the previous theorem
\begin{proof}
    Consider the prime decomposition os $m= p_1^{a_1}p_2^{a_2}\ldots p_k^{a_k}$ and $n= q_1^{b_1}q_2^{b_2}\ldots q_l^{b_l}$ where $p_1,p_2,\ldots,p_k,q_1,q_2,\ldots,q_l$ are distinct primes and $a_1,a_2,\ldots,a_k,b_1,b_2,\ldots,b_l$ are positive integers given that $\gcd(m,n)=1$.

    Any divisor $d$ of $mn$ has a subset of $\{p_1,\ldots,p_k,q_1,\ldots,q_l\}$ as its prime factors. \begin{itemize}
        \item Collect the $p_i$'s into a set $A$ and let the product of the elements in $A$ be $d_1$.
        \item Collect the $q_i$'s into a set $B$ and let the product of the elements in $B$ be $d_2$.
    \end{itemize}
    Now $d=d_1 d_2, d_1|m, d_2|n$. Moreover, $d_1$ and $d_2$ are relatively prime, uniquely determined by $d$. 

    Then $F(mn)=\sum_{d|mn}f(d)=\sum_{d_1|m,d_2|n}f(d_1d_2)=\sum_{d_1|m,d_2|n}f(d_1)f(d_2)=\sum_{d_1|m}f(d_1)\sum_{d_2|n}f(d_2)=F(m)F(n)$.

    It shows that $F(n)$ is multiplicative.
\end{proof}
\begin{corollary}
    The following two arithmetic functions are multiplicative:

    $\sigma(n) = $ sum of all positive divisors of $n$.

    $\tau(n) = $ number of positive divisors of $n$.
\end{corollary}
Example: $n=100=2^2\times 5^2$. 

Then $\sigma(100)=\sigma(2^2)\sigma(5^2)=(1+2+4)(1+5+25)=\frac{2^3-1}{2-1}\frac{5^3-1}{5-1}=217$.

$\tau(100)=\tau(2^2)\tau(5^2)=3\times 3=9$.
\begin{proof}
    If $f(n)=1$, then $F(n)=\sum_{d|n}f(d)=\tau(n)$.

    If $f(n)=n$, then $F(n)=\sum_{d|n}d=\sigma(n)$.
\end{proof}
Knowing that $F(n)$ is multiplicative, we can write $F(n)=\prod_{i=1}^k F(p_i^{r_i})$ where $n=p_1^{r_1}p_2^{r_2}\ldots p_k^{r_k}$ is the prime factorization of $n$.

This is by using the multiplicative property of $F(n)$ and the prime factorization of $n$.

It leads to the following formula:\[
    \tau(n)=\prod_{i=1}^k \tau(p_i^{r_i})=(r_1+1)(r_2+1)\ldots (r_k+1)
\]
This is because $\tau(p_i^{r_i})=r_i+1$, which is the number of divisors of $p_i^{r_i}$. Divisors of $p_i^{r_i}$ are $1,p_i,p_i^2,\ldots,p_i^{r_i}$.
\subsection{Perfect numbers}
\begin{theorem}
    Let $n$ be a positive integer. Then $n$ is a perfect number if and only if $\sigma(n)=2n$.

    Equivalently, $n$ is the sum of its proper divisors.
\end{theorem}
For example: $n=6$ is perfect because $\sigma(6)=1+2+3+6=12=\frac{2^2-1}{2-1}=\frac{3^2-1}{3-1}=2\times 6$.Or the other way: the proper divisors of 6 are 1,2,3. Then $1+2+3=6$.
\begin{theorem}
    Suppose $m\ge 2$ is a prime and $p=2^m-1$ is also a prime. Then $n=2^{m-1}p$ is a perfect number.
\end{theorem}
\begin{proof}
    $\sigma(n)=\sigma(2^{m-1})\sigma(p)=\frac{2^m-1}{2-1}\frac{p^2-1}{p-1}=(2^m-1)(p+1)=2^m p=2n$
\end{proof}
\begin{lemma}
    If $2^m-1$ is prime, then $m$ is prime.
\end{lemma}
\begin{proof}
    If $m=ab$ is composite, $a,b\ge 2$, then $2^m-1=2^{ab}-1=(2^a)^b -1 =(2^a-1)((2^a)^{b-1}+(2^a)^{b-2}+\cdots+2^a+1)$ is also composite.
\end{proof}
However, it does not work the other way around. If $m$ is prime, $2^m-1$ may or may not be prime.

In fact only 51 \textbf{Mersenne prime} is known, largest known is $2^{82589933}-1$.
\begin{align*}
    p = 2 &\implies 2^2-1=3 & \text{Mersenne prime}\\
    p = 3 &\implies 2^3-1=7 & \text{Mersenne prime}\\
    p = 5 &\implies 2^5-1=31 & \text{Mersenne prime}\\
    p = 7 &\implies 2^7-1=127 & \text{Mersenne prime}\\
    p = 11 &\implies 2^{11}-1=2047=23\times 89 & \text{Composite}\\
    p = 13 &\implies 2^{13}-1=8191 & \text{Mersenne prime}\\
    p = 17 &\implies 2^{17}-1=131071 & \text{Mersenne prime}\\
    p = 19 &\implies 2^{19}-1=524287 & \text{Mersenne prime}\\
    p = 23 &\implies 2^{23}-1=47\times 178481 & \text{Composite}\\
    p = 29 &\implies 2^{29}-1=233\times 1103\times 2089 & \text{Composite}\\
    p = 31 &\implies 2^{31}-1=2147483647 & \text{Mersenne prime}
\end{align*}
\begin{theorem}
    Even numbers of the form $n=2^{m-1}(2^m-1)$ where $2^m-1$ is prime are perfect numbers.

    Prime numbers of the form $2^m-1$ are called \textbf{Mersenne primes}.
\end{theorem}
\subsubsection{Euler's Perfect Number Theorem}
\begin{theorem}
    Let $n$ be an even perfect number, then there exists a prime number $m\ge 2$ such that $n=2^{m-1}(2^m-1)$ and $2^m-1$ is prime.
\end{theorem}
\begin{proof}
    Write $n=2^s t$ where $t$ is odd. Our goal is to show that $t=2^{s+1}-1$ is prime. Note that if we set $m=s+1$, then $n=2^{m-1}t$ and $2^m-1=t$. We can do it since $n$ is even, $s \ge 1$ and $t$ is odd.

    Since $n$ is perfect, then $\sigma(n)=2n$. Then $\sigma(n)=\sigma(2^s)\sigma(t)=(2^{s+1}-1)\sigma(t)=2n=2^{s+1}t$. 

    We conclude that $2^{s+1} | \sigma(t)$ since $2^{s+1}-1$ and $2^{s+1}$ are relatively prime. Then $\sigma(t)=2^{s+1}q$ for some $q\in \mathbb{N}$.

    Substitute it in we obtain $(2^{s+1}-1)q=t$. Thus $1\le q<t, q|t$. 

    Claim that $q=1$. If $q>1$, then $t$ has a proper divisor $q$ such that $q<t$. Then $\sigma(t)\ge 1+q+t=2t+q>2t$. On the other hand, $\sigma(t)=2^{s+1}q=q+t$ since $(2^{s+1}-1)q=t$. Then we have a contradiction.

    Therefore $q=1$ and $t=2^{s+1}-1$ is prime. Moreover, $\sigma(t)=2^{s+1}q=2^{s+1}=t+1$

    So $t$ is a prime. In Summary, $n=2^{s+1}-1$ is prime and $n=2^{s+1}(2^{s+1}-1)$.
\end{proof}
Euler's Theorem shows that the problems of finding all even perfect numbers and finding all Mersenne primes are equivalent.
\subsubsection{Mobius Inversion Formula}
Let $f(n)$ be a multiplicative function. We know that $F(n)=\sum_{d|n}f(d)$ is also multiplicative. Can we recover $f(n)$ from $F(n)$?

Mobius function: $\mu(n)=\begin{cases}
    0 & \text{if } n \text{ has a square factor} \\
    (-1)^k & \text{if } n \text{ is a product of } k \text{ distinct primes}
\end{cases}$

In particular, $\mu(1)=1$, since 1 is a product of 0 distinct primes.

Let the prime decomposition of $n$ be $n=p_1^{a_1}\cdots p_r^{a_r}$, then $\mu(n)=0$ if $\exists a_i\ge 2$ for some $i$ and $\mu(n)=(-1)^r$ if $n$ is a product of $r$ distinct primes where $a_i=1$ for all $i$.

Mobius Inversion Formula: \[
    f(n)=\sum_{d|n}F(d)\mu(\frac{n}{d})
\]
For example: $f(n)=1$ for every $n$, then $F(n)=\sum_{d|n}1=\tau(n)$ which is the number of positive divisors of $n$. By the inversion formula, $\sum_{d|n}\tau(d)\mu(\frac{n}{d})=1$.

Similarly, taking $f(n)=n$, we obtain $F(n)=\sum_{d|n}d=\sigma(n)$. It is the sum of the positive divisors of $n$. Then $\sum_{d|n}\sigma(d)\mu(\frac{n}{d})=n$.
\begin{lemma}
    $\mu(n)$ is multiplicative.
\end{lemma}
\begin{proof}
    Let $n=p_1^{a_1}p_2^{a_2}\ldots p_k^{a_k}$ and $m=q_1^{b_1}q_2^{b_2}\ldots q_l^{b_l}$ where $p_1,p_2,\ldots,p_k,q_1,q_2,\ldots,q_l$ are distinct primes and $a_1,a_2,\ldots,a_k,b_1,b_2,\ldots,b_l$ are positive integers. We can do that since $\gcd(m,n)=1$.

    Then $\mu(nm)=\begin{cases}
        0 & \text{if } n \text{ has a square factor} \\
        (-1)^k & \text{if } n \text{ is a product of } k \text{ distinct primes}
    \end{cases}$

    Where is my real proof?

    Then $\mu(nm)=\mu(n)\mu(m)$.
\end{proof}
\begin{lemma}
    If $n$ is a positive integer, then \[
        \sum_{d|n}\mu(d)=\begin{cases}
            1 & \text{if } n=1 \\
            0 & \text{if } n>1
        \end{cases}
    \]
\end{lemma}
\begin{proof}
    Let $M(n)=\sum_{d|n}\mu(d)$. Then $M(1)=\mu(1)=1$. 

    Note that $M(n)$ is multiplicative.

    If $n=p_1^{a_1}p_2^{a_2}\ldots p_k^{a_k}$, then $M(n)=\sum_{d|n}\mu(d)=\prod_{i=1}^k M(p_i^a)$%=\prod_{i=1}^k \sum_{a=0}^{a_i}\mu(p_i^a)$.

    Observe that for any $M(p^a)=\sum_{d|p^a}=1+(-1)+0+\cdots+0=0$ if $a>1$ and $M(p)=1-1=0$ if $a=1$.

    Then $M(n)=0$ if $n>1$.
\end{proof}

Now come back to the Mobius Inversion Formula. We can write it as \[
    f(n)=\sum_{d|n}F(d)\mu(\frac{n}{d})=\sum_{d|n}F(\frac{n}{d})\mu(d)
\]
\begin{proof}
    The idea is to change the order of summation. Note that $F(\frac{n}{d})=\sum_{e|\frac{n}{d}}f(e)$.

    Then \[
        f(n)=\sum_{d|n}F(\frac{n}{d})\mu(d)=\sum_{d|n}\sum_{e|\frac{n}{d}}f(e)\mu(d)=\sum_{e|n}\sum_{d|\frac{n}{e}}f(e)\mu(d)=\sum_{e|n}f(e)\sum_{d|\frac{n}{e}}\mu(d)
    \]
\end{proof}
\subsection{Primality Testing}
Given a positive integer $n$, how can we determine whether $n$ is prime?
\begin{itemize}
    \item factoring means $n$ into a product of $ab$ where $1<a,b<n$.
    \item testing $n$ for primality means checking whether $n$ has any divisors other than 1 and $n$. In other words, check whether or not a factorization of $n$ exists.
\end{itemize}

There are fast tests for primality but no fast factorization algorithms are known.

This discrepancy is the foundation for RSA cryptography. 
\subsubsection{Fermat's Primality Test}
One way to know if a number is not prime is to find a witness to its compositeness.

If there exisits an integer $b$ in $[1,n-1]$ such that $b^{n-1}\not\equiv 1 \pmod n$, then $n$ is not prime.

We can do it fast since we can do repeating squaring to find $b^{n-1}$ in $O(\log n)$ time.

Note that we cannot compute $b^{n-1}$ using Euler's Theorem since we do not know the prime factorization of $n$.

So if $b$ exists we know something, but we cannot find it. 

Example: $n=15,b=2$. Then $2^{15-1}=2^{14}\equiv 2^{4\times 3 +2}\equiv 16^3 \times 2^2 \equiv 4 \pmod {15}$. Then 15 is not prime.

Claim: $n$ is \textbf{pseudo-prime} to base $b$ if $b^{n-1}\equiv 1 \pmod n$.

A pseudo prime of 1 base does not imply it is a prime for all bases. For example, 341 is a pseudo prime to base 2, but it is not a prime.\[
    2^{341-1}\equiv 2^{340}\equiv 1 \pmod {341}
\]
while $341=11\times 31$.

\begin{proof}
    By CRT, we can show that $\begin{cases}
        2^{341-1}\equiv 1 \pmod {11} \\
        2^{341-1}\equiv 1 \pmod {31}
    \end{cases}$
\end{proof}
By Fermat's Little Theorem:\begin{align*}
    2^{10}\equiv 1 \pmod {11} & \implies 2^{340}\equiv 1 \pmod {11} \\
    2^{5}\equiv 1 \pmod {31} & \implies 2^{340}\equiv 1 \pmod {31}
\end{align*}

We can find that 341 is not a pseudo prime to base 3.\[
    3^{340}\equiv (3^{30})^{11} \times 3^{10}\equiv 1^{11}\times 3^{10}\equiv 3^{10}\equiv {3^3}^3\times 3\equiv (-4)^3\times 3\equiv 25 \pmod {341}
\]

Procedure for Fermat's Primality Test:\begin{enumerate}
    \item Choose a random integer $b$ in $[2,n-2]$.
    \item Compute $b^{n-1}\pmod n$.
    \item If $b^{n-1}\not\equiv 1 \pmod n$, then $n$ is not prime.
    \item If $b^{n-1}\equiv 1 \pmod n$, then $n$ is a pseudo prime to base $b$.
    \item Repeat the test with a different $b$.
    \item If $n$ is a pseudo prime to enough bases, then $n$ is \textbf{probably} prime.
    \item It is possible that $n$ is a pseudo prime to all bases and still not prime. Such numbers are called \textbf{Carmichael numbers}.
\end{enumerate}
Note that it is a probabilistic test instead of an extensive test. It is not possible to go through all the bases to test the primality of a number.
\subsubsection{Carmichael Numbers}
\begin{proposition}
    Suppose $n=p_1 p_2 \cdots p_r$ where $p_1,p_2,\ldots,p_r$ are distinct primes. If $p_i -1\mid n-1$ for all $i$, then $n$ is a Carmichael number.
\end{proposition}
\begin{proof}
    Let $b$ be any integer not divisible by $p_i,\forall i$. In other words, $\gcd(b,n)=1$. Then by Fermat's Little Theorem, $b^{p_i-1}\equiv 1 \pmod {p_i}$. Then $b^{n-1}\equiv 1 \pmod {p_i}$.

    Then $b^{n-1}\equiv 1 \pmod {p_i}$ for all $i$. 
    
    Let $x=b^{n-1}-1$. Then $x\equiv 1 \pmod {p_i},\forall i$. By Chinese remainder theorem, $x\equiv 1 \pmod n$. Then $b^{n-1}\equiv 1 \pmod n$.
    
    Then $n$ is a Carmichael number.
\end{proof}
For example $561=3\times 11\times 17$. Then $3-1,11-1,17-1\mid 561-1$. Then 561 is a Carmichael number.

Another example: $6601=7\times 23\times 41$. Then $7-1,23-1,41-1\mid 6601-1$. Then 6601 is a Carmichael number.
\subsubsection{Miller-Rabin Primality Test}
The Miller-Rabin Primality Test is a more efficient probabilistic primality test than Fermat's Primality Test. It is a refined version of Fermat's Primality Test which captures the idea of Carmichael numbers.

Say we want to test $n$ for primality. \begin{enumerate}
    \item Start out the same way as Fermat's Primality Test. Choose a random integer $b$ in $[2,n-2]$. Then compute $b^{n-1}\pmod n$.
    \item If $b^{n-1}\not\equiv 1 \pmod n$, then $n$ is not prime.
    \item If $b^{n-1}\equiv 1 \pmod n$, then we probe a bit deeper by computing $x\equiv b^{(n-1)/2}\pmod n$. (Here we are assuming that $n-1$ is even. If $n-1$ is odd, then $n$ is not prime since $b\ge 2$)
    \item Note that $x^2=b^{n-1}\equiv 1 \pmod n$. If $n$ is a prime, then $x\equiv \pm 1 \pmod n$. 
    \item If $x\equiv 1 \pmod n$, and $\frac{n-1}{2}$ is odd, We cannot say anything here, pick another $b$ and repeat the test.
    \item If $x\equiv -1 \pmod n$, then $n$ is \textbf{probably} prime. We cannot say anything here, pick another $b$ and repeat the test.
    \item However, if $x\equiv 1 \pmod n$, and $\frac{n-1}{2}$ is even, then we can dig deeper by computing $y\equiv b^{(n-1)/4}\pmod n$.
    \item Same idea: If $y\equiv - 1 \pmod n$, or $y\equiv 1 \pmod n$ and $\frac{n-1}{4}$ is odd, then pick another $b$ and repeat the test.
    \item If $y\equiv 1 \pmod n$, and $\frac{n-1}{4}$ is even, then repeat checking. 
    \item The only stop criterion is when we conclude that $n$ is composite.
    \item We never conclude with 100\% certainty that $n$ is prime. We can only say that $n$ is \textbf{probably} prime. It is a probabilistic test!
\end{enumerate}
Use the smallest Carmichael Number 561 again. Perform the Miller-Rabin Primality Test with $b=5$: \begin{itemize}
    \item $5^{560}\equiv 1 \pmod {561}$.
    \item $5^{280}\equiv 67 \pmod {561}$. (By performing repeating squaring, a pain to do but the easiest method we can find already)
\end{itemize}
Note that $561 = 3 \times 11 \times 17$. \begin{itemize}
    \item Assume $\gcd(b,561)=1$. 
    \item $b^{280}\equiv (b^2)^{140}\equiv 1 \pmod 3$ 
    \item $b^{280}\equiv (b^{10})^{28}\equiv 1 \pmod {11}$
    \item $b^{280}\equiv b^8 \pmod {17}$
    \item When $b=5$, $5^{280}\equiv 5^8 \equiv 25^4\equiv 8^4\equiv 64^2\equiv (-4)^2\equiv 16\equiv -1 \pmod {17}$
    \item Now with CRT: $\begin{cases}
        x\equiv 1 \pmod 3 \\
        x\equiv 1 \pmod {11} \\
        x\equiv -1 \pmod {17}
    \end{cases}
    \implies x\equiv 67 \pmod {561}$
\end{itemize}
\subsubsection{Rabin's Theorem}
\begin{theorem}
    Fix a composite integer $n$. Pick $a,b\in [2,n-2]$ at random. Then the probability that $n$ fails Miller's test is greater than or equal to $\frac{1}{4}$.

    In other words, Miller's test will detect that $n$ is composite with probability at least $\frac{1}{4}$ for all possible choices
\end{theorem}
Equivalently, if we run Miller's test $k$ times, with $k$ different choices of $b$, and $n$ passes each of the test, then the probability that $n$ is composite is less than or equal to $(\frac{3}{4})^k$.
\subsection{Pollard's Factorization Method}
Goal: Factor a given (large) integer $n$.

\begin{enumerate}
    \item Choose $r_0\equiv 2^{0!}\pmod n$.
    \item Compute the next one using the formula $r_k\equiv r_{k-1}^{k} \pmod n$.
    \item For each $k$ compute $\gcd(r_{k}-1,n)=g_k$. Note that $1\le g_k \le r_{k-1} \le n-2$ and $g_k$ divides $n$.
    \item For each $k$  So if $g_k > 1$, then we have found a factor of $n$ which is greater than 1.
    \item Repeat the process till we find a factor of $n$.
\end{enumerate}
The idea here is that if $n$ has a prime factor $p$ such that $p-1$ divides $k!$, then $2^{k!}\equiv 1 \pmod p$. By Fermat's theorem,  $2^{k!}\equiv 1 \pmod n$ and $g_k>1$. 

In ohter words, p divides both $r_{k}-1$ and $n$ implies that $p$ divides $g_k = \gcd(r_{k}-1,n)$. 

The method works well if $p-1 \mid k!$ for some prime factor $p$ of $n$ for small $k$. Informally, this means that $p-1$ has ``small'' prime factors. Thus the weakness of this algorithm is that we do not know which $k$ will work.

\subsubsection{Example}
Factor 689 using Pollard's method:\begin{align*}
    r_1 & \equiv 2 \pmod {689} && \gcd(r_1-1, 689)=1 \\
    r_2 & \equiv 2^2 \equiv 4 \pmod {689} && \gcd(r_2-1, 689)=1 \\
    r_3 & \equiv 4^3 \equiv 64 \pmod {689} && \gcd(r_3-1, 689)=1 \\
    r_4 & \equiv 64^4 \equiv 66 \pmod {689} && \gcd(r_4-1, 689)=13 \\
    \gcd(65,689) & = 13 && 689=13\times 53
\end{align*}
Factor 10403: \begin{align*}
    &r_2\equiv 2^2\equiv 4 \pmod {10403} && \gcd(r_2-1,10403)=1 \\
    &r_3\equiv 4^3\equiv 64 \pmod {10403} && \gcd(r_3-1,10403)=1 \\
    &r_4\equiv 64^4\equiv 7580 \pmod {10403} && \gcd(r_4-1,10403)=1 \\ 
    &r_5\equiv 7580^5\equiv 4438 \pmod {10403} && \gcd(r_5-1,10403)=1 \\
    &r_6\equiv 4438^6\equiv 6862 \pmod {10403} && \gcd(r_6-1,10403)=1 \\
    &r_7\equiv 6862^7\equiv 137 \pmod {10403} && \gcd(r_7-1,10403)=1 \\
    &r_8\equiv 137^8\equiv 196 \pmod {10403} && \gcd(r_8-1,10403)=1 \\
    &r_9\equiv 196^9\equiv 3619 \pmod {10403} && \gcd(r_9-1,10403)=1 \\
    &r_{10}\equiv 3619^{10}\equiv 9798 \pmod {10403} && \gcd(r_{10}-1,10403)=101 \\
    &\gcd(9799,10403)  = 101 && 10403=101\times 103
\end{align*}
$p=101$ is a prime factor of 10403. Then $p-1=100$. Then $100=2^2 5^2\mid 10!$. Then $p-1\mid 10!$ when $k=10$.

\section{Cryptography}
A cryptosystem consists of:\begin{itemize}
    \item A set `P' of possible \textbf{plaintexts}. (unencrypted messages)
    \item A set `C' of possible \textbf{ciphertexts}. (encrypted messages)
    \item An enciphering function/transmission $f$ that maps $P$ into $C$. $f$ should be a bijection.
    \item A deciphering function $g$ that maps $C$ into $P$. $g$ should be the inverse of $f$. $g\circ f (x)=x$ for all $x\in P$.
    \item P and C usually consist of integers or congruence classes mod n for some n.
\end{itemize}

In a \textbf{classical cryptography} system, both $f$ and $f^{-1}$ are unknown to the public. The security of the system depends on the secrecy of $f$ and $f^{-1}$. Knowing one can get the other.

In a \textbf{public key cryptosystem}, $f$ is public and $f^{-1}$ is private. The security of the system depends on the difficulty of computing $f^{-1}$ from $f$.

First example: $P=C=\{0,1,\ldots,25\}=\mathbb{Z}\pmod {26}$. This corresponds to the 26 letters of the English alphabet.

$f(x)=x+b \pmod{26}$ for some $b\in \mathbb{Z}\pmod {26}$. $b$ is the key. It is called shift cipher, $b$ is both the encryption and decryption key.

\subsection{Classcial Cryptography}
\subsubsection{Shift Cipher}
Consider the shift cipher $P=C=\mathbb{Z}_n$
\begin{align*}
    f(x)=x+b \pmod n && \text{Encryption key b} \\
    g(x)=x-b \pmod n && \text{Decryption key b}
\end{align*}
Since there are 26 letters in the English alphabet, we can set $n=26$.

$P=C=\mathbb{Z}_{26}= \{0,1,\ldots,25\}$. Then $f(x)=x+b \pmod {26}$ and $g(x)=x-b \pmod {26}$.
\subsubsection{Breaking cryptosystems}
To break a cryptosystem, one needs two types of information:\begin{itemize}
    \item The general nature of the system: $P,C,f,g$.
    \item The specific key used in the system.
\end{itemize}
We usually assume that the cryptanalyst knows the general nature of the system but not the specific key.

One way of breaking shift cipher is to use frequency analysis. The frequency of letters in English text is not uniform. For example, the letter `e' is the most common letter in English text.

Let's say we are given a ciphertext ``FQOCUDEM'', then we try encryption function $f:E \to$ for each of the letters in the text. \begin{align*}
    f:E\to U \implies b=16 && \text{FQOCUDEM}\to \text{PAYMENOW }
\end{align*}
\subsubsection{Affine Cipher}
The affine cipher is a generalization of the shift cipher. It is a type of monoalphabetic substitution cipher. It is a combination of shift and multiplication.

$P=C=\mathbb{Z}_{26}$. Let $a,b\in \mathbb{Z}_{26}$ with $\gcd(a,26)=1$. Then $f(x)=ax+b \pmod {26}$ and $g(x)=a^{-1}(x-b) \pmod {26}$.

Note that $a^{-1}$ exists if and only if $\gcd(a,26)=1$. To solve for $a^{-1}$, we can use the Euclidean algorithm and back substitution. $as+a^{-1}t\equiv 1 \pmod {26}$. 

To break it, we can try to use frequency analysis again: 
\begin{align*}
    f(x)=ax+b \pmod {26} && g(x)=a^{-1}(x-b) \pmod {26}
\end{align*}
For example, let's break a ciphtertext where the most frequent letters are `K' and `D'. In English, the most frequent letters are `E' and `T'. Then we can try to find $a,b$ such that $f(K)=E$ and $f(D)=T$.

Now we need to solve the system of equations: \begin{align*}
    g:K(10)\to E(4) && 10c+d\equiv 4 \pmod {26} \\
    g:D(3)\to T(19) && 3c+d\equiv 19 \pmod {26}
\end{align*}
Want to solve the system of linear congruence for $c,d$.

Subtract both equations to eliminate $d$ to get $7c\equiv 11 \pmod {26}$. Then $c\equiv 7^{-1}\cdot 11 \equiv 9 \pmod {26}$. Then $d\equiv 4-10\cdot 9 \equiv 20 \pmod {26}$.

Then $f(x)=9x+20 \pmod {26}$ and $g(x)=15(x-20) \pmod {26}$.

Another example: $n=28$. Elements corresponds to $\mathbb{Z}_{28}$ where $0\to 25 \implies$ `A'$\to$`Z', $26 \to$ ` ' (space), $27\to$ `?'.

Suppose the most frequent letters are `B' and `?'. By frequency analysis, most frequent should be ` ' and `E'. 

So $1\to 26, 27\to 4$. We need to solve for \begin{align*}
    c + d &\equiv 26 \pmod {28} \\
    27c+d &\equiv 4 \pmod {28}
\end{align*} 
Subtract both sides, we get $26c\equiv -22 \pmod {28}$. Then $2c\equiv 22 \pmod {28}$. Note that $\gcd(2,22)=2$ so 2 does not have an inverse modulo 28. We have two possible solutions for it. 

Solve for $c\equiv 11 \pmod {\frac{28}{22}}$ to get $c_1 = 11, c_2 = 11+14 = 25$. Then $d_1\equiv 26-11 \equiv 15 \pmod {28}$ and $d_2\equiv 26-25 \equiv 1 \pmod {28}$.

We try both to see which works better. 
\subsection{Exponentiation Cipher}
Here $P=C=\mathbb{Z}_p$ where $p$ is a prime number. 
\begin{itemize}
    \item Encryption: $f(x)=x^e \pmod p$ where $e$ is the encryption key.
    \item Decryption: $g(x)=x^d \pmod p$ where $d$ is the decryption key.
    \item Note the range is $\mathbb{Z}_p$.
\end{itemize}
This requires that $g(f(x))=x$ for all $x\in \mathbb{Z}_p$. For example, $(x^e)^d \equiv x^{ed} \equiv x \pmod p$.

\begin{lemma}
    $de\equiv 1 \pmod {p-1} \implies g(f(x))=x$ for all $x\in \mathbb{Z}_p$.
\end{lemma}

\begin{proof}
    It follows one of the corollaries from Fermat's Little Theorem. $m\equiv n \pmod {p-1} \implies x^m\equiv x^n \pmod p$.
\end{proof}

Note that it requires that $e,d$ are inverses modulo $p-1$. To make sure that an inverse exists, we need to make sure that $\gcd(e,p-1)=1$. Then we can get $d= e ^{-1} \pmod {p-1}$.

For example, $p=29$, $e=5$. Then $f(x)=x^5 \pmod {29}$ and $g(x)=x^{17} \pmod {29}$ given by: \begin{align*}
    5a &\equiv 1 \pmod {28}\\
    17 &\equiv 5^{-1} \pmod {28} \text{ by Euclidean algorithm}
\end{align*}

Then decoding can be done by repetitively squaring. For example, to decode $f(3)=3^5 \pmod {29}$, we can do $3^5\equiv 3^{1+4} \equiv 3\cdot 3^4 \equiv 3\cdot 81 \equiv 243 \equiv 3 \pmod {29}$.
\subsubsection{Breaking Exponentiation Cipher}
Note that exponentiation cipher is slightly safer than affine cipher. It is not as easy to break by frequency analysis.

To break it, we need to solve congruences of the form $y^d \equiv x \pmod p$ for $y$. If $x,y \in \mathbb{R}$, $d$ can be recovered by taking the discrete logarithm of $x$ to the base $y$. This is called the discrete logarithm problem.

For large $p$, the problem is computationally difficult. For example, if $p$ is a 1000-bit prime, then the discrete logarithm problem is computationally infeasible.
\subsection{RSA Cryptosystem}
RSA is a public key cryptosystem. It was invented by Rivest, Shamir, and Adleman in 1977. It is based on the difficulty of factoring large integers.

Each user chooses two large primes $p$ and $q$ and an encryption exponent $e$. The public key is $(n,e)$ where $n=pq$ and $e$ is the encryption exponent. Here we need $\gcd(e,\phi(n)) = 1$ where $\phi(n)=(p-1)(q-1)$.The private key is $(n,d)$ where $d$ is the decryption exponent.
\begin{itemize}
    \item Encryption: $f(x)=x^e=y \pmod n$.
    \item Decryption: $g(y)=y^d=x \pmod n$. $d = e^{-1} \pmod {\phi(n)}$ where $\phi(n)=\phi(p)\phi(q)=(p-1)(q-1)$. 
\end{itemize}
\begin{theorem}
    $x^{ed}\equiv x \pmod n$ for all $x\in \mathbb{Z}_n$.
\end{theorem}
The difference between RSA and the exponentiation cipher is that $n$ is not prime. The security of RSA is based on the difficulty of factoring large integers.

Note that Euler's Theorem can only handle some cases since $x$ might not be relatively prime to $n$.

To prove the theorem, we need to show that $ed\equiv 1 \pmod {\phi(n)}$. Then $x^{ed}\equiv x \pmod n$.

Need to show that $\begin{cases}
    y\equiv x \pmod p \\
    y\equiv x \pmod q
\end{cases}$. If we can do that then by CRT, we can show that $y\equiv x \pmod n$.\begin{proof}
    Given that $ed\equiv 1 \pmod {\phi(n) = (p-1)(q-1)}$. Then $\begin{cases}
        ed\equiv 1 \pmod {p-1} \\
        ed\equiv 1 \pmod {q-1}
    \end{cases} $
    Apply Corollary 2 of Fermat's theorem. $ed\equiv 1 \pmod {(p-1)(q-1)}$. Then $\begin{cases}
        ed\equiv 1 \pmod {p-1} \\
        ed\equiv 1 \pmod {q-1}
    \end{cases}  \implies \begin{cases}
        x^{ed}\equiv x^1 \pmod {p} \\
        x^{ed}\equiv x^1 \pmod {q}
        \end{cases}$

    Now apply CRT, we can show that  $x^{ed}\equiv x \pmod n$.
\end{proof}
For example: $p=281, q=167, n = pq = 46927$. Let $(n,e) = (46927, 39423)$ as the public key. Transmit blocks of 3 letters: there are 26 letters, then there are $26^3 = 17576$ possible blocks. 

If the message is ``YES'', which corresponds to 24, 4, 18. View them as a three digit number $(24, 4, 18)_{26}$ which is $24\cdot 26^2 + 4\cdot 26 + 18 = 16346$. 

Encode will get me $16346^{39423} \pmod {46927} = 21166$ using repetitively squaring. Now separate the letter into base 26: $21166=1\cdot 26^3 + 5\cdot 26^2 + 8\cdot 26 + 2$. Then decode it to get ``BFIC''.

To decipher, first need to find $d\equiv e^{-1} \pmod {\phi(n)}$. Here $\phi(n) = (p-1)(q-1) = 280\cdot 166$. Using the Euclidean algorithm we can find $d = 26767$. Then $21166^{26767} \pmod {46927} = 16346$. Then decode it to get ``YES''.

Note that we can solve it here since we know both $p$ and $q$. In practice, we do not know $p$ and $q$.
\subsection{Implementing RSA}
\subsubsection{Find large primes}
To implement RSA, we need to find large primes. We can use the Miller-Rabin primality test to find large primes.

To get a prime about 200 digits long, we can pick integer x between 1 and $10^{200}$ and test if $x$ is prime. If it is not, then we can try another x. According to the prime number theorem, the number of finding a prime is about $\frac{10^{200}}{\ln (10^{200})}$. This probability is about $\frac{1}{\ln(10^{200})}=\frac{1}{200\ln(10)}\approx \frac{1}{460}$. And we know we will only choose $x$ that is not divisible by $2,3,5,7,11$ since these can be found to be obviously wrong. So the probability is even higher.

Note that $p,q$ should not be close to each other because of Fermat's factorization method. If $p,q$ are close, then $n$ is small and can be factored easily. Also, each $p-1. q-1$ should have large prime factors because of Pollard's factorization method.
\subsubsection{Choosing encryption exponent}
The encryption exponent $e$ should be relatively prime to $\phi(n)$. Although we can choose at random and it might work, it is better to choose another large prime that is larger than $p$ and $q$.
\subsubsection{Procedure}
\begin{enumerate}
    \item Find large primes $p$ and $q$ in 200 digits range by Miller-Rabin primality test. Set $n=pq$.
    \item Choose an encryption exponent $e$ that is relatively prime to $\phi(n) = (p-1)(q-1)$. Now the public key is $(n,e)$.
    \item Use Euclidean Algorithm to find $d$ such that $ed\equiv 1 \pmod {\phi(n)}$. Now the private key is $(n,d)$. (Solve for $ex+\phi(n)y=1$, then $d\equiv x \pmod {\phi(n)}$)
    \item Encrypt the message by $f(x)=x^e \pmod n$.
    \item Decrypt the message by $g(y)=y^d \pmod n$. Use repeated squaring to compute $y^d$.
\end{enumerate}
\subsection{Attacks on RSA}
\subsubsection{Fermat's Factorization Method}
Want to factor $n$: 

Let $s$ range over the integers $> \sqrt{n}$ until $s^2-n$ is a square. Then $s^2-n=t^2$. Then $s^2-t^2=n$. Then $(s+t)(s-t)=n$. Then $n$ is a product of $s+t$ and $s-t$. $s = \frac{p+q}{2}$ and $t = \frac{p-q}{2}$.

Number of steps to find $s$ and $t$ is about $s-\sqrt{n} = \frac{p+q}{2} - \sqrt{pq} = \frac{p-2\sqrt{pq}+q}{2} = \frac{(\sqrt{p}-\sqrt{q})^2}{2}$. $s$ is small if $p$ and $q$ are close. But if $p>>q$, then $\approx \sqrt{p}$ steps are needed.

For example: Factor 5959 using Fermat's factorization method.\begin{enumerate}
    \item Find that $77<\sqrt{5959}<78$. Then $78\leq s$.
    \item $s = 78 \implies s^2 - n = 78^2 - 5959 = 125$. Not a square.
    \item $s = 79 \implies s^2 - n = 79^2 - 5959 = 282$. Not a square.
    \item $s = 80 \implies s^2 - n = 80^2 - 5959 = 441 = 21^2$. Then $s=80, t=21$.
    \item Then $5959 = 80^2 - 21^2 = 101\times 59$.
    \item Then $p=101, q=59$.
\end{enumerate}
Thus to keep RSA secure, $p$ and $q$ should be not close to each other. A good strategy might be to choose them a few digits apart.

One can also try to factor $n=pq$ using Pollard's factorization method. To be safe, need at least one of $p-1,q-1$ to have a large prime factor.
\subsubsection{Chose ciphertext attack}
We use the fact that: \[
    (x_1 x_2)^e \equiv x_1^e x_2^e \pmod n \quad (y_1 y_2)^d \equiv y_1^d y_2^d \pmod n
\]

Suppose we want to decrypt a cipher text message $M \pmod n$. We do not know the decryption key $d$. Choose a mask $r$ such that $\gcd(r,n)=1$. Now we need to convince the holder of the private key to decrypt $M\cdot r^e \pmod n$. Note that $e$ is public.

Decrypted message is $(M\cdot r^e)^d \equiv M^d r \pmod n$. Now we can recover $M$ by multiplying by $r^{-1} \pmod n$. Note how we choose $r$ such that $\gcd(r,n)=1$.
\subsubsection{Small exponent attack}
If $e$ is small, then the encryption function $f(x)=x^e \pmod n$ can be broken by brute force. We can try to compute $\sqrt[e]{y*} \pmod n$ where $y* = x^e + kn \pmod n$ for some $k\in \mathbb{Z}$.

If $e$ is small, there are a few $k$ we need to try to find the correct $x$. For example, if $e=3$, then we only need to try $k=0,1,2$ to find the correct $x$.
\end{document}