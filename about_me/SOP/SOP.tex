\documentclass[a4 paper, 10pt]{article}
\usepackage{graphicx,xcolor} % Required for inserting images

\definecolor{mylinks}{RGB}{21, 92, 78} % you can change the color from here

\usepackage[T1]{fontenc}
\usepackage{tgbonum}
\usepackage[top=2cm,bottom=2cm,left=1.5cm,right=1.5cm]{geometry}
\usepackage{fontawesome5}
\usepackage{tabularx,tabulary,multirow}

\usepackage[colorlinks=true,urlcolor=mylinks,linkcolor=mylinks
]{hyperref}
\usepackage[skip=10pt plus1pt, indent=40pt]{parskip}
\usepackage{blindtext}
\usepackage{hyperref}
% \usepackage{etoolbox}

% I used this guy's template
% \title{SoP_Template}
% \author{Mayukh Chakrabarty}
% \date{29 September 2023}



\begin{document}
\newcommand{\ifstringequal}[4]{%
  \ifnum\pdfstrcmp{#1}{#2}=0
  #3%
  \else
  #4%
  \fi
}


\newcommand{\schoolName}{UBC}
\newcommand{\theSchoolFullName}{University of British Columbia}
\csdef{path-UBC}{bank/ubc.tex}
\newcommand{\ccc}{bank/ubc.tex}



\newcommand{\theDepartment}{Electrical and Computer Engineering}

% \ifstrequal{\schoolName}{UBC}{
    
%     \ifstrequal{\schoolName}{U of T}{
%         \input{uoft_file.tex}
%     }{
%         % Default case or other cases
%         % \input{default_file.tex}
%     }
% }

\pagestyle{empty} % suppresses page numbers

%header
{\fontfamily{qbk}\selectfont % font can be changed from https://www.overleaf.com/learn/latex/Font_typefaces

\begin{center}
    \begin{minipage}{.9\textwidth}
        \Large{\textbf{STATEMENT OF PURPOSE}}    
    \end{minipage}

    \begin{tabularx}{.8\textwidth}{X r X}
    Xingyu Wang (Tom)                 & \multicolumn{2}{r}{\faIcon{mobile} +1 604-388-5164}     \\
        BASC in Computer Engineering  & \multicolumn{2}{r}{\faIcon{envelope}  \href{tomxingyuwang@gmail.com}{tomxingyuwang@gmail.com}} \\
        University of British Columbia  &  \multicolumn{2}{r}{\faIcon{linkedin} \href{www.linkedin.com/in/tom-wang-554904220/}{linkedin-profile}}
    \end{tabularx}
    \par\noindent\rule{\textwidth}{1.25pt}
\end{center}

}

{\fontfamily{ptm}\selectfont % font can be changed from https://www.overleaf.com/learn/latex/Font_typefaces

This Staement of Purpose is for the application to the graduate program in department of \theDepartment{} at \theSchoolFullName{}.

\section*{Interests of study}
So far, the charm of \theDepartment{} in my opinion is how I can make computations faster and cheaper, whether it is through software optimization or hardware resources. Nowadays as the demand of computations grows since the evolution of AI, the need of faster and cheaper computations is more urgent than ever.

I am particularly interested in the field of computer architecture and operating systems, which are the two main components that determine computational resources and how to utilize it. Any improvements in these two fields can lead to a significant increase in computational power, which is the key to the development of AI and other computational-heavy applications.

With the school background in UBC, I have been actively involved in the research of cache and page prefetching with Machine Learning, which, if implemented correctly, can significantly reduce the time of memory access and decrease the computation unit idle time. The detials of this can be found \hyperref[prefetching]{below}.

There are a few area that I would like to explore in the future:\begin{itemize}
    \item \textbf{Cache and Memory Optimization:} The cache and memory are the two main components that determine the speed of computation. I would like to explore how to optimize the cache and memory whether it is through hardware(Memory design etc.) or software(prefetching or eviction policy, management, etc.).
    \item \textbf{Operating System Optimization:} The operating system is the bridge between the hardware and software. I would like to explore how to optimize the operating system to make the computation faster and cheaper. It could be through the scheduling policy, memory management, etc.
    \item \textbf{Parallel Computing:} The parallel computing is the future of computation. I would like to explore how to utilize the parallel computing.
    \item \textbf{Machine Learning in Computer Architecture:} Machine Learning is a powerful tool that can be used in computer architecture, possibly to optimize branch prediction, task scheduling, coherence protocol, etc.
    \item \textbf{And the list goes on...} There are areas that I have not explored yet like in memory processing, quantum computing, etc. I would like to explore these areas as well.
\end{itemize}
\section*{Future Goals}
Right now I am more leaning towards the academia, to be added...
\section*{Why \theSchoolFullName{}?}

\ifdefstring{\schoolName}{U of T}{\input{bank/ut.tex}}{}
\ifdefstring{\schoolName}{UBC}{I have been in UBC for 4 years and I still love the school and the courses offered here.

I am familiar with the campus, the professors, the courses, and the research projects. I have been involved in a research project with Prof. Alexandra Fedorova for the past year and I would like to continue the research with the professors that I know, and the professors that know me.

I have already taken few grad level course in ECE and I really enjoyed the courses. I would like to take more grad level courses in the future.

The research experience I had with Prof. Alexandra Fedorova has been great. I have learned a lot from the project and I would like to continue the research in the future.}{}

\section*{Motivation}
When I first entered the university, I was not exposed of any details of how computations are done,
\section*{My Background}
During the 4-year-undergraduate study, I spent all my time at UBC.
\subsection*{Academic Background}
I took various courses in computer engineering, electrical engineering and computer science at UBC. 

\begin{itemize}
    \item For hardware, I took courses like Digital Logic Design, Computer Architecture, VLSI Design, Error control coding etc. 
    \item For software, I took courses like Operating Systems, Software construction, Data Structure, Algorithm, Machine Learning etc.
\end{itemize}

All the above courses have given me a solid foundation in both hardware and software, which I believe is essential for the graduate study in \theDepartment{} at \theSchoolFullName{}.
\subsection*{Awards and Scholarships}
\begin{enumerate}
    \item I have received the Dean's Honour List for 3 years in a row, which is the recognition of my academic achievement at UBC.
    \item I also received the NSERC Undergraduate Student Research Award at the third year summer, which encouraged me to pursue research in the field of computer engineering. It is a funding that provides 4-month minimum wage for undergraduate students to work on research projects.
\end{enumerate}

\subsection*{Work Experience}
\label{prefetching}
The work started in the third year summer where I received NSERC awards provided by Prof. Alexandra Fedorova. I was hired as a full-time research assistant to work on the project of cache and page prefetching with Machine Learning. 

The goal of the project is to reduce the time of memory access and decrease the computation unit idle time. The project is still ongoing and I am still actively involved in the project.

\subsubsection*{The motivation of the project}
The motivation of the project is that the memory access is the bottleneck of the computation.

The memory access time is much slower than the computation unit. If we can reduce the time of memory access, we can significantly reduce the idle time of the computation unit, which can lead to a significant increase in the computation speed.

Privious work has shown that LSTM model could significantly reduce the amount of cache misses. We wanted to extend that to solve page faults since disk access is much slower than memory accesses.

\subsubsection*{The progress of the project}
The first 2 months was getting familiar with the project, tools, and the previous work. 

To be continued...

}
\end{document}
